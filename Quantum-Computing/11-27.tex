\section*{Thanksgiving Special!!!}
We'll be doing a bit of review/Q and A today as a result of the large number of abscences for Thanksgiving.

Question: Can you go over homework 5, looking at the Schmidt decomposition/POVMs

Answer: Everything we do in quantum mechanics is about making predictions. Measurements will be described by some set of POVMs. This replaces the old system of projectors. Recall:
\begin{align*}
	\sum_n \Pi_n &= 1 &
	\Pi_n^\dagger &= \Pi_n &
	\Pi_n \geq 0
\end{align*}
Question two was of course not about prediction, but rather about retrodiction. We look at a state with classical randomness, being in either state $\ket{1}$ or $\ket{-}$.
By using a set of three POVMs rather than a standard basis of 2 measurements we can get an improvement in distinguishability between the two states.

Question: Are the set of Kraus operators the same for completely positive maps and for POVMs?

Answer: They are the same set. This can be seen because the following relations apply to Kraus operators:
\begin{align*}
	\sum_n K_n^\dagger K_n &= 1 & 
	K_n\rho K_n^\dagger \geq 0
\end{align*}
This will imply all the properties of the POVMs. Additionally of note is the fact that a POVM doesn't have a unique Kraus Operator representation, and these different representations will give different states after measurement.

Question: Problem 5, looking at the Schmidt decomposition of:
\begin{align*}
	\ket{\psi} &= \frac{\ket{00} + \ket{01} + \ket{10} - \ket{11}}{2}
\end{align*}

Answer: We want to look at the density matrix of each state seperated out, so:
\begin{align*}
	\rho_A &= \frac{1}{2}\ket{+}\bra{+} + \frac{1}{2}\ket{-}\bra{-} \\
	\rho_A &= \unit \\
	\rho_B &= \unit
\end{align*}
So this is a maximally entangled state. We pick our basis in $A$ as $\ket{+},\ket{-}$:
\begin{align*}
	\bra{+}\ket{\psi} &= \frac{1}{\sqrt{2}}\ket{0} \\
	\bra{-}\ket{\psi} &= \frac{1}{\sqrt{2}}\ket{1}
\end{align*}
So our Schmidt decomposition is then:
\begin{align*}
	\ket{\psi} &= \frac{1}{\sqrt{2}}\ket{+}\ket{0} + \frac{1}{\sqrt{2}}\ket{-}\ket{1}
\end{align*}
The size of the Schmidt decomposition can never be larger than the dimension of the smaller Hilbert space.

Question: What will the final look like?

Answer: It'll be looking at where errors happen in the quantum computers hosted by qBraid. The general idea is to run n $X$ gates to see how the error rate scales with the number of gates.
Odd and even gates may have differing behavior.
