These measures of entanglement are desirable because they seem to match the Von Neumann entropy for the singlet state. Entanglement in the modern view is considered a resource with which we can bypass some of our normal physical rules.
We say that we restrict ourselves to LOCC, which corresponds to allowing only Local qubit Operations, and Classical Communication.
In this model our actors are restricted, if Alice and Bob are our actors, then if they start with no entanglement, they can only produce seperable states.
A result of analyzing this system is that entanglement of some state $\rho_{AB}$, cannot increase on average in the confines of LOCC.
This puts restrictions on any measure of entanglement that we have access to. This can be terribly difficult to prove for a given measurement of entanglement. We also impose the condition that the entanglement of a Bell state should be 1.
We can also use the number of terms in our Schmidt decomposition as a good measure of entanglement.

We now want to define a physical map with the following properties:
\begin{align*}
	\rho' &= \mathcal{L}(\rho) &
	\rho'\ ^\dagger &= \rho' \\
	\rho' &\geq 0 &
	\Tr{\rho'} &= 1 \\
	\mathcal{L}(\alpha\rho_1 + \beta\rho_2) &= \alpha \mathcal{L}(\rho_1) + \beta\mathcal{L}(\rho_2) &
	\forall H' \land \tilde{\rho} \in (H' \otimes H); (\mathcal{L} \otimes \unit)(\tilde{\rho}) &\geq 0
\end{align*}
For an operator $\mathcal{L}$ which satisfies all these conditions we call it a completely positive map.

For an example we have: \\
\begin{quantikz}
	\lstick{$\ket{\psi}$} & \gate[2]{U} & \rstick{$\rho_0 = \Tr_{q_1}(\ket{\psi}\bra{\psi})$} \\
	\lstick{$\ket{0}$} & &
\end{quantikz} \\
This clearly must correspond to some physical operation. If we choose for our example: \\
\begin{quantikz}
	\lstick{$\ket{\psi}= \alpha\ket{0} + \beta\ket{1}$} & \ctrl{1} & \rstick{$\rho_0 = \Tr_{q_1}(\ket{\psi}\bra{\psi})$} \\
	\lstick{$\ket{0}$} &\targ{} &
\end{quantikz} \\
Where our output state will be:
\begin{align*}
	\rho_0 &= \alpha^2\ket{0}\bra{0} + \beta^2\ket{1}\bra{1} 
\end{align*}
We can then see that our operation can be written as (potentially):
\begin{align*}
	\mathbb{L}(\rho) &= P_0 \rho P_0 + P_1 \rho P_1
\end{align*}
In general any completely positive map will be of the form:
\begin{align*}
	\mathcal{L}(\rho) &= \sum_k A_k \rho A_k^\dagger &
	\sum_k A_k^\dagger A_k &= 1
\end{align*}
This is known as the Kraus representation of the completely positive map, where the $A_k$ are the Kraus operators.
Physically for our prior example these operators seem to correspond to taking a measurement of our system.

If we look at the Schmidt decomposition of our state:
\begin{align*}
	\rho &= \sum_i p_i P_i \\
	\rho &= \sum_i \sqrt{p_i} P_i \sqrt{p_i}
\end{align*}
So our Karus operators can be thought of as changing both the states and the probabilities:
\begin{align*}
	A_k \sqrt{p_i} \ket{\psi} &= \sqrt{q_{i,k}}\ket{\phi_{i,k}}
\end{align*}

For example we look at a set of Kraus operators:
\begin{align*}
	A_1 &= \begin{pmatrix}
		1 & 0 \\
		0& \sqrt{1-p}
	       \end{pmatrix} \\
	A_2 &= \begin{pmatrix}
		0 & \sqrt{p} \\
		0 & 0
	       \end{pmatrix}
\end{align*}
These add up to the identity, so these correspond to a physical map. Physically this seems to correspond to spontaneous decay, where $A_1$ leaves $\ket{0}$ unchanged, and reduced $\ket{1}$ by a factor of $1-p$, and $A_2$ turns $\ket{1}$ to $\ket{0}$ with a factor of $p$.

Another example is:
\begin{align*}
	A_1 &= \sqrt{p} X \\
	A_2 &= \sqrt{q} Y \\
	A_3 &= \sqrt{r} Z \\
	A_4 &= \sqrt{1-p-q-r}\unit
\end{align*}
Where $A_1,A_2,A_3$ represent potential errors that occur, and $A_4$ represents the system proceeding without any error.
\section{Physical Implementations of Quantum Computers}
\subsection{Overview}
We can see that all atoms are pretty similar to the Hydrogen atom (with some minor differences). We will begin with these familiar systems, looking at neutral atoms, and ions.
We will then take a look at doing superconducting quantum computers. We won't cover quantum dots, NV centers, or optical quantum computers.
\subsection{Physical Qubit}
These are typically very small systems, that are typically manipulated via lasers. For more man-made systems (quantum dots, Superconducting qubits), we have considerably larger qubits, which are manuplulated with different tools.

One advantage of the natural qubits, is that all the systems are identical, while the manmade systems, while each qubit is very similar, they have small differences.
On the other hand the man-made qubits are typically in fixed, known locations, whereas for natural systems, we need to worry about where the atoms are in space.
