In order to now construct gates, we consider interacting our laser field with our qubit. If the field interacts, rather than is there simply as a reference, we can then create a new hamiltonian:
\begin{align*}
	\mathcal{H} &= \hbar\begin{pmatrix}
		0 & \frac{\Omega}{2} e^{i\omega t} \\
		\frac{\Omega}{2} e^{-i\omega t} & \omega_r
		       \end{pmatrix}
\end{align*}
Where $\Omega$ is the Rabi frequency and $\omega$ is the laser's frequency. In order to deal with our Hamiltonian's explicit time dependance, we move to the field interaction picture, so:
\begin{align*}
	\ket{\psi(t)} &= \tilde{c}_0 \ket{0} + \tilde{c}_1 e^{-i\omega t}
\end{align*}
Which changes our Hamiltonian to:
\begin{align*}
	\tilde{\mathcal{H}} &= \hbar\begin{pmatrix}
		0 & \frac{\Omega}{2} \\
		\frac{\Omega}{2} & \omega_r-\omega
		       \end{pmatrix}
\end{align*}
So:
\begin{align*}
	\ket{\psi(t)} &- e^{-i\tilde{\mathcal{H}} t/\hbar}\ket{\psi(0)} \\
	\ket{\psi(t)} &- U(t)\ket{\psi(0)} \\
\end{align*}
So then if $\omega_r = \omega$ (on resonance):
\begin{align*}
	U(t) &= \begin{pmatrix}
		\cos\frac{\Omega t}{2} & -i\sin\frac{\Omega t}{2} \\
		i\sin\frac{\Omega t}{2} & \cos\frac{\Omega t}{2}
		\end{pmatrix}
\end{align*}
If we pick an even more special case $\Omega t = \frac{\pi}{2}$, then:
\begin{align*}
	U\left(\frac{\pi}{2\Omega}\right)  &= \frac{1}{\sqrt{2}}\begin{pmatrix}
		1 & -i \\
		-i & 1
					      \end{pmatrix}
\end{align*}
Which is close to a Hadamard gate. This works for any pulse with ``pulse area'' $A = \int dt \Omega(t) = \frac{\pi}{2}$.
If we instead choose a $\pi$ pulse $A = \pi$, so:
\begin{align*}
	U(A=\pi) &= \begin{pmatrix}
		0 & -i \\
		-i & 0
		    \end{pmatrix}
\end{align*}
If we then look at a $2\pi$ pulse, we find:
\begin{align*}
	U(A=2\pi) &= -\unit
\end{align*}
Which is a global phase away from the $x$ gate.
By applying this to a transition between our state and an ancilla level of our system, then this will be equivalent to a $z$ gate. I.e:
\begin{align*}
	U(A_{12}=2\pi) &= \begin{pmatrix}
		1 & 0 & 0 \\
		0 & -1 & 0 \\
		0 & 0 & -1
			       \end{pmatrix}
\end{align*}
Where the part of the system we care about is the $z$ gate.

We now look at a far detuned laser $|\Delta| \gg \Omega$. We look for the eigenvalues:
\begin{align*}
	\det \begin{pmatrix}
		-\lambda & \frac{\Omega}{2} \\
		\frac{\Omega}{2} -\Delta -\lambda
	     \end{pmatrix} &= 0 \\
	     -\lambda(-\lambda -\Delta) -\frac{\Omega^2}{4} &= 0 \\
	     \lambda_\pm &= \frac{-\Delta \pm \sqrt{\Delta^2 + \Omega^2}}{2}
\end{align*}
Expanding the eignevalue that is close to zero:
\begin{align*}
	\lambda_+ &=-\frac{\Delta}{2} + \frac{\Delta}{2} \sqrt{1 + \frac{\Omega^2}{\Delta^2}} \\
	\lambda_+ &=-\frac{\Delta}{2} + \frac{\Delta}{2} \left(1 + \frac{1}{2}\frac{\Omega^2}{\Delta^2}\right) \\
	\lambda_+ &\approx\frac{\Delta}{2}\frac{1}{2}\frac{\Omega^2}{\Delta^2} \\
	\lambda_+ &\approx\frac{\Omega^2}{4\Delta}
\end{align*}
This energy shift can be used to give a state an arbitrary phase shift. This is known as the AC stark shift.

When working with neutral atoms, we want to drive transitions to higher states. These are typically transitions from S to P, so this is governed by an electric dipole interaction, which has Hamiltonian:
\begin{align*}
	H_\text{int} &= -\bm{d}\cdot\bm{E} \\
	\bm{d} &= -e\bm{r}
\end{align*}
If we pick our $\bm{E}$ field to face along the $z$ direction (essentially vertically polarized light):
\begin{align*}
	\bm{E} &= E \hat{e}_z
\end{align*}
This laser drives a transition from $\ket{2}$ to $\ket{e'}$ and we also seperately drive a transition from $\ket{1}$ to $\ket{e}$.
\begin{align*}
	H_\text{int} &= eE\bm{r}\cdot\hat{e}_z \\
	H_\text{int} &= eEz \\
	\frac{1}{2}\Omega_1 &= \bra{1}ezE\ket{e'} \\
	\frac{1}{2}\Omega_0 &= \bra{0}eZE\ket{e}
\end{align*}
We know that $\ket{0}$ has even z parity, therefore in order to have a non-zero $\Omega_0$ we need $\ket{e}$ to have odd z parity, and since $\ket{1}$ has odd z parity $\ket{e'}$ must have even z parity. \\
If these are far detuned then this will not actually cause transitions, but instead will lead to energy shifts to $\ket{0}$ and $\ket{1}$.
These cause spatially varying potentials, which allow you to trap neutral atoms. A typically trapping time can be as much as 100 seconds.

In order to do two qubit gates we need a mechanism to cause neighboring qubits to interact with eachother. To do this we use Rydberg sttes, which are states with a very high $n$ quantum number (electrons are very far away from the atom here because $r \propto n^2$).
In order to make this a stable-ish state we choose the state with the highest angular momentum $m=n-1$. Although this is very hard to drive, it can be done in multiple steps, or with a very strong laser.
We choose these states to be the right distance, so if two neighboring atoms are both in Rydberg states they will interact, but if either isn't they won't.

We perform this operation by first doing a $\pi$ pulse to transfer the first qubit from $\ket{1}$ to $\ket{r}$ (if the qubit is in $\ket{0}$ this will not drive any transition).
Then we attempt to apply a $2\pi$ pulse to the second qubit from $\ket{1}$ to $\ket{r}$, but if the first qubit is in $\ket{r}$ we will shift the energy and the transition will not occur.
Finally we apply another $\pi$ pulse to the first qubit to return from $\ket{r}$ to $\ket{1}$. Therefore we have the following mappings:
\begin{align*}
	\ket{00} &\to \ket{00} \\
	\ket{11} &\to -\ket{11} \\
	\ket{01} &\to -\ket{01} \\
	\ket{10} &\to -\ket{10}
\end{align*}
This is enough, with our single qubit operations, to construct a CNOT gate, and thus we have all we need for our quantum computer.
