We have a state structure for the Hydrogen atom of 1 ground state and 4 excited states. If we incoorporate the fine structure interaction (electron spin), we have more states without energy spitting, i.e. we have 2 s state and 6 p state.
If we instead look at the hyperfine structure (nuclear spin), which splts us to have 4 ground states that have an energy splitting, where one is lower than the other three. The lowest state is the singlet state, while the other three at triplet states.

Spins will only directly interact with magnetic fields, rather than electric fields. This is a dipole interaction. This interaction is related to the intrinsic angular momentum of our particle, which generally can be represented by a vector of operators:
\begin{align*}
	\bm{J} &= \begin{pmatrix}
		J_x \\
		J_y \\
		J_z
		  \end{pmatrix}
\end{align*}
And in general for angular momenta we know:
\begin{align*}
	[J_i,J_j] &= i\hbar J_k \epsilon_{ijk}
\end{align*}
Which matches the properties of classical angular momenta. 

From these it follows that $[J_i,J^2] = 0$. This implies that overall angular momenta has simultaneous eigenvalues with any individual angular momentum. Therefore:
\begin{align*}
	J_z \ket{m,J} &= m\hbar\ket{m,J} \\
	J^2\ket{m,J} &= \hbar^2 J(J+1)\ket{m,J} \\
	m &\in [-J, -J +1,\ldots, J-1, J]
\end{align*}
Clearly this can only be satisfied if $J \in \{\frac{m}{2} :m\in \mathbb{N}\}$. Classically we can add angular momenta simply, but for quantum mechanical angular momenta we need to instead add these according to:
\begin{align*}
	\bm{J} &= \bm{J}_1 \otimes \unit_2 + \unit_1 \otimes \bm{J}_2
\end{align*}
For this new quantity to be another angular momentum we need our relations to still hold. We know from classical mechanics that:
\begin{align*}
	|J_1 - J_2| \leq J \leq J_1 + J_2
\end{align*}
And as before we need our values of $J$ to be stepped in integers.

For our system we have a hamiltonian proportional to:
\begin{align*}
	\mathcal{H} &\propto \bm{S}_\text{electron} \cdot\bm{S}_\text{nucleous}
\end{align*}
For eigenstates we need the spins to be aligned/antialigned. We can find the eigenvalues of $(\bm{S}_\text{electron} + \bm{S}_\text{nucleous})^2 = \bm{S}^2_\text{electron} + \bm{S}^2_\text{nucleous} + 2\bm{S}_\text{electron}\cdot\bm{S}_\text{nucleous}$. 
Since both the electron and the nucleous have spin $\frac{1}{2}$ we can say $S_\text{electron} = S_\text{nucleous} = \frac{1}{2}$ and $S = \{0,1\}$. Therefore we can say(ignoring factors of $\hbar^2$):
\begin{align*}
	S(S+1) &= \frac{3}{2} + 2\bm{S}_\text{electron}\cdot\bm{S}_\text{nucleous} \\
\end{align*}
Which gives us the two posibilites:
\begin{align*}
	\bm{S}_\text{electron}\cdot\bm{S}_\text{nucleous} &= -\frac{3}{4} \\`
	\bm{S}_\text{electron}\cdot\bm{S}_\text{nucleous} &= \frac{1}{4}
\end{align*}
We can quickly see that the following states must correspond to $S=1$ (higher energy):
\begin{align*}
	\ket{\psi} &= \ket{\uparrow_\text{el} \uparrow_\text{nu}} & m &= 1 \\ 
	\ket{\psi} &= \ket{\downarrow_\text{el}\downarrow_\text{nu}} & m &= -1
	\sqrt{2}\ket{\psi} &= \ket{\uparrow_\text{el} \downarrow_\text{nu}} + \ket{\downarrow_\text{el} \uparrow_\text{nu}} & m &= 0
\end{align*}
And the $S=0$ is then:
\begin{align*}
	\sqrt{2}\ket{\psi} &= \ket{\uparrow_\text{el} \downarrow_\text{nu}} - \ket{\downarrow_\text{el} \uparrow_\text{nu}} & m &= 0
\end{align*}

These triplet states will decay to the singlet state by emitting a photon with $\lambda \approx 21\text{cm}$. The decay time for a single atom is on the order of $10^7$ years.
Unfortunately Hydrogen is too light for practical purposes, so we typically use heavier atoms. A common example is Cesium, which has one valence electron, so is somewhat analagous to the hydrogen atom.
Our hyperfine splitting will be quite different here since the nuclear spin for cesium. Cesium has an overall nuclear spin of $I = \frac{7}{2}$ (for the most commonly used isotope).
Therefore our overall spin will be $F=\{3,4\}$. For each our $z$ spin will have a wide range, if $F=3$ we have 7 states, and if $F=4$ we have 9 states. These states are seperated by a $9.1926 \text{GHz}$ transition.
This transition is used to define the second.

In order to remove the large amount of ambiguity between this states, we apply a magnetic field to our system to split the energy levels for all the different states.
In order to keep our system robust against variations in the magnetic field we want to use the $m=0$ states.

Looking only at our two level system, in the case of free evolution, we have:
\begin{align*}
	\mathbb{H} &= \hbar \begin{pmatrix}
		\omega_0 & 0 \\
		0 & \omega_1
			    \end{pmatrix}
\end{align*}
We can see for a superposition input state we have:
\begin{align*}
	\ket{\psi(0)} &= c_0\ket{0} + c_1\ket{1} \\
	\ket{\psi(t)} &= e^{-i\omega_0 t} \left(c_0\ket{0} + e^{-i(\omega_1-\omega_0)t} c_1\ket{1}\right)
\end{align*}
We can see that this state is rapidly changing. If we add in a reference laser:
\begin{align*}
	\forall n: \ket{0}\ket{n} \to \ket{1}\ket{n-1}
\end{align*}
If our reference laser is tuned to our transition than the phase applied by this reference laser perfectly cancels the phase from the transition.
