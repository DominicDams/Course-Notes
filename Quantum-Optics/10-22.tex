\subsection{Denstiy Matrix as an ensemble state}
We define an ensemble state in terms of a mixture of density operators:
\begin{align*}
	\hat{\rho} &= \frac{1}{N} \sum_i \hat{\rho}_i
\end{align*}
Where each $\hat{\rho}_i$ is a pure state. This should evolve acording to:
\begin{align*}
	i\hbar \partial_t \hat{\rho} &= [H,\hat{\rho}]
\end{align*}
This should behave much like we have seen before. With the sort of decay and dephasing we expect for a two level system.
\subsection{Summary}
In this section we have used $\rho$ to describe and atom. We focused on reduced density operators to describe states where we don't have total information. We also used mixed density operators to describe ensembles of atoms.

We used off diagonal elements to describe our coherences of our states.

We also described our system using Bloch vectors, and the evolution using Bloch equations.
\section{Fields in two level systems}
\subsection{Review of Maxwell's equations}
We consider Maxwell's equations in a dielectric (no free charge or current, but not vaccuum):
\begin{align*}
	\del\cdot\bm{D} &= 0 &
	\del\cdot\bm{B} &= 0 \\
	\del\cross\bm{E} &= -\partial_t\bm{B} &
	\del\cross\bm{H} &= -\partial_t\bm{D}
\end{align*}
Where we have:
\begin{align*}
	\bm{D} &= \epsilon_0\bm{E} + \bm{P} &
	\bm{H} &= \frac{1}{\mu_0}\bm{B} - \bm{M}
\end{align*}
For a dielectric $\bm{M} = 0$ so $\bm{H} = \frac{1}{\mu_0}\bm{B}$ \\
We now derive the wave equations:
\begin{align*}
	\del\cross(\del\cross\bm{E}) &= -\partial_t\del\cross\bm{B} \\
	\del(\del\cross\bm{E}) - \del^2\bm{E}  &= -\mu_0\partial_t^2\bm{D}
\end{align*}
If we have no spatial variation in electric succeptibility $\del(\del\cross\bm{E}) =0$, so:
\begin{align*}
	\del^2\bm{E}  &= \mu_0\partial_t^2\bm{D} \\
	\left(\del^2 - \frac{1}{c^2}\right)\bm{E} &= \mu_0\partial_t^2\bm{P}
\end{align*}
Which is a wave equation with a sourcing term determined by $\bm{P}$. We know that $\bm{P} = \frac{N}{V} \expval{\mu}$ which is the density of dipoles. We can then write this as:
\begin{align*}
	\bm{P} &= \frac{N}{V} \Tr{\hat{\rho}\bm{\mu}} \\
	\bm{P} &= \frac{N}{V} \Tr{\rho_{12}\bm{\mu}_{12} + \text{c.c.}}
\end{align*}
Where we've dropped the diagonal terms since $\bm{\mu}$ has $0$ for diagonal elements. \\
\subsection{Slowly varying amplitude and phase approximation}
We now make approximations to convert our second order PDE into a first order PDE.\\
We assume we're interacting with a single mode optical field. This mode is a plane wave polarized along $\hat{e}_x$ and propogating along $\hat{e}_z$. So:
\begin{align*}
	\bm{E}(\bm{r},t) &= \frac{1}{2}\hat{e}_xE_0(z,t)e^{-i[\omega t-zk-\phi(z,t)]} + \text{c.c.} \\
	\bm{E}(\bm{r},t) &= \bm{E}_+ + \bm{E}_-
\end{align*}
We then say our polarization is:
\begin{align*}
	\bm{P}(\bm{r},t) &= \frac{1}{2}\hat{e}_xP_0(z,t)e^{-i[\omega t-zk-\phi(z,t)]} + \text{c.c.} \\
	\bm{P}(\bm{r},t) &= \bm{P}_+ + \bm{P}_-
\end{align*}
Where $E_0$ is real and $P_0$ is complex. Additionally $\frac{\omega}{k} = c$ \\
We say this is slowly varying so these variables remain nearly unchanged during one spatial or temporal period. I.e.:
\begin{align*}
	|T\partial_t E_0 | \ll |E_0| &
	|\frac{1}{E_0}\partial_t E_0 | \ll \omega  \\
	|\lambda \partial_z E_0| \ll |E_0| &
	|\frac{1}{E_0} \partial_z E_0| \ll k \\
	|\frac{1}{\phi}\partial_t\phi | \ll \omega &
	|\frac{1}{\phi}\partial_z\phi | \ll k
\end{align*}
We now use the trick that:
\begin{align*}
	\left(\partial_z^2 - \frac{1}{c^2}\partial_t^2\right) &= \left(\partial_z + \frac{1}{c}\partial_t\right)\left(\partial_z - \frac{1}{c}\partial_t\right)
\end{align*}
So we then say:
\begin{align*}
	\left(\partial_z - \frac{1}{c}\partial_t\right)\bm{E}_+ &= \left(ik + \frac{i\omega}{c}\right)\bm{E}_+ + O(\dot{\bm{E}_0}, \dot{\bm{P}_0},\ldots) \\
	\left(\partial_z - \frac{1}{c}\partial_t\right)\bm{E}_+ &\approx 2ik\bm{E}_+ \\
	\left(\partial_z + \frac{1}{c}\partial_t\right)\bm{E}_+ &= e^{i\phi}E_0 \left(\partial_z + \frac{1}{c}\partial_t\right)e^{i(\omega t - kz)} + \ldots (\text{Product rule})
\end{align*}
For the polarization:
\begin{align*}
	\partial_t^2\bm{P}_+ &\approx -\omega^2\bm{P}_+
\end{align*}
If we work through the math we find:
\begin{align*}
	\left(\partial_z + \frac{1}{c}\partial_t\right)E_0 + iE_0\left(\partial_z + \frac{1}{c}\partial_t\right)\phi &= -\frac{\omega^2}{2ik}\mu_0P_0 \\
	\left(\partial_z + \frac{1}{c}\partial_t\right)E_0 + iE_0\left(\partial_z + \frac{1}{c}\partial_t\right)\phi &= -\frac{k}{2i\epsilon_0}P_0
\end{align*}
Looking at this in terms of the real and imaginary parts:
\begin{align*}
	\left(\partial_z + \frac{1}{c}\partial_t\right)E_0 &= -\frac{k}{2\epsilon_0} \Im P_0 \\
	E_0\left(\partial_z + \frac{1}{c}\partial_t\right)\phi &= \frac{k}{2\epsilon_0} \Re P_0 \\
\end{align*}
Therefore we can say the Imaginary part of $P_0$ determines our absorption/gain, and the real part determines the dispersion.\\
We now imagine our system, as an electric field incident on a ensemble of a large number of two level systems, where the ourput field is modified by the two level systems. \\
We know that our polarization is generated by our electric field, so we can write this diagramatically as:
\begin{align*}
	\bm{E} \to_{\text{Bloch}} \bm{P} \to_{\text{Maxwell}} \bm{E} \to_\text{Bloch} \ldots
\end{align*}
So these connections must be self-cosnsitent. \\
\subsection{Linear absorption/dispersion}
This is the simplest form we can imagine, where $P_0 \propto E_0$. We also assume that this isotropic so $\bm{P}_0\propto\bm{E}_0$. \\
We start by defining a linear susceptibility:
\begin{align*}
	P_0 &= \epsilon_0\chi(\omega)E_0
\end{align*}
Where then:
\begin{align*}
	\chi(\omega) &= \chi'(\omega) + i\chi''(\omega)
\end{align*}
So $\chi'$ describes dispersion and $\chi''$ describes absorption. \\
We can now solve our first order differential equations:
\begin{align*}
	\left(\partial_z + \frac{1}{c}\partial_t\right)E_0 &= -\frac{k}{2} \chi'' E_0 \\
	E_0\left(\partial_z + \frac{1}{c}\partial_t\right)\phi &= \frac{k}{2} \chi' E_0
\end{align*}
Our steady state solutions(i.e. the behavior after the system has settled so $\partial_t E_0 =0$) are then:
\begin{align*}
	\partial_z E_0 &= -\frac{k}{2}\chi''E_0
\end{align*}
If we say $\alpha = k\chi''$ and $I \propto E_0^2$, then:
\begin{align*}
	\partial_z I &= -\alpha I &
	I &= I_0 e^{-\alpha z}
\end{align*}
For our dispersion we can then see:
\begin{align*}
	\left(\partial_z + \frac{1}{c}\partial_t\right)\phi &= \frac{k}{2} \chi'
\end{align*}
Looking at the steady state again $\partial_t\phi =0$:
\begin{align*}
	\partial_z\phi &= \frac{k}{2} \chi' &
	\phi &= \frac{k}{2} \chi'z 
\end{align*}
Now looking at how this impacts the overall phase we see:
\begin{align*}
	e^{-i[\omega t- kz -\phi]} &= e^{-i\left[\omega t -k\left(1+\frac{\chi'}{2}\right)z\right]}
\end{align*}
And so we can then see our phase velocity is then:
\begin{align*}
	\nu_{ph} &= \frac{\omega}{k\left(1+\frac{x'}{2}\right)} \\
	\nu_{ph} &= \frac{c}{1+\frac{x'}{2}} \\
	\nu_{ph} &= \frac{c}{\eta} \\
	\eta &= 1 \frac{\chi'}{2}
\end{align*}
Which defines our index of refraction in terms of the susceptibility. \\
In order to determine $\chi$ we need to calculate $\rho_{12}$ which requires us to solve the optical Bloch equation:
\begin{align*}
	\dot{\tilde{\rho}}_{21} &= -\gamma \tilde{\rho}_{21} -i\delta\tilde{\rho}_{21} + i\frac{\Omega_0}{2}(\rho_{22} - \rho_{11})
\end{align*}
Since this is in a steady state we can say $\tilde{\rho}_{21} = 0$. Also since this is linear we can say $\rho_{21} \propto E_0$. So we want only the first order solution in $E_0$:
\begin{align*}
	\dot{\tilde{\rho}}_{21}^{(1)} &= -\gamma \tilde{\rho}_{21}^{(1)} - i\delta\tilde{\rho}_{21}^{(1)} + i\frac{\Omega_0}{2}(\rho_{22}^{(0)} - \rho_{11}^{(0)}) \\
	0 &= -\gamma \tilde{\rho}_{21}^{(1)} - i\delta\tilde{\rho}_{21}^{(1)} + i\frac{\Omega_0}{2}(\rho_{22}^{(0)} - \rho_{11}^{(0)}) \\
	\tilde{\rho}_{21}^{(1)} &= \frac{\Omega_0}{2} \frac{\rho_{22}^{(0)} - \rho_{11}^{(0)}}{\delta - i\gamma}
\end{align*}
