We now look at $\partial_z A_l$:
\begin{align*}
	\partial_z A_l &= \int \frac{d\omega}{2\pi} \left[\partial_z \tilde{A} + i\tilde{A}(k(\omega)-k_l)\right] e^{i[(k(\omega)-k_l)z - (\omega - \omega_l)  t]} \\
	&= S_{NL} + S_L \\
	S_{NL} &=  \int \frac{d\omega}{2\pi} \frac{i\omega}{2\epsilon_0 c n(\omega)} e^{-ikz} \tilde{P}_{NL}e^{i[(k(\omega)-k_l)z - (\omega - \omega_l)  t]} \\
	&=  \int \frac{d\omega}{2\pi} \frac{i\omega}{2\epsilon_0 c n(\omega)} e^{-ik_lz} \tilde{P}_{NL}e^{-i(\omega - \omega_l) t} \\
	&\approx  \frac{i\omega_l}{2\epsilon_0 cn(\omega_l} e^{-ik_l z}\int \frac{d\omega}{2\pi} \tilde{P}_{NL} e^{-i(\omega-\omega_l)t} \\
	&\approx  \frac{i\omega_l}{2\epsilon_0 cn(\omega_l} e^{-ik_l z} P_l^{NL}(z,t) \\
	S_L &= \int \frac{d\omega}{2\pi} i\tilde{A}(k(\omega)-k_l) e^{i[(k(\omega)-k_l)z - (\omega - \omega_l)  t]} \\
	&\approx \int \frac{d\omega}{2\pi} i\tilde{A}k_l'(\omega-\omega_l) e^{-i(\omega - \omega_l)  t} e^{i(k-k_l)z} \\
	&\approx -k_l'\partial_t\int \frac{d\omega}{2\pi} \tilde{A}e^{-i(\omega - \omega_l)  t} \\
	&\approx -\frac{1}{v_{gl}}\partial_t A_l \\
	\partial_z A_l +\frac{1}{v_{gl}} \partial_t A_l &= \frac{i\omega_l}{2\epsilon_0 n(\omega_l)} e^{-ik_lz} P_l^{NL}
\end{align*}
This equation will describe the evolution of our fields from a non-linear interaction.

\subsection{Example: 3 wave mixing}
We start by saying our nonlinear term is caused by a non-zero $\chi^{(2)}$, looking in the band description of the fields, we can write:
\begin{align*}
	E &= \sum_l A_l e^{i(k_l z - \omega_l t)} \\
	E^2 &= \sum_{lm} A_lA_m^* E^{i(k_l-k_m)z}e^{-i(\omega_l - \omega_m) t}
\end{align*}
(Note we have terms with all combinations of complex conjugation in the second line)

We can describe SFG in terms of three bands where $\omega_1 + \omega_2 = \omega_3$, so an input fields in $\omega_1$ and $\omega_2$ causes the generation of a field at $\omega_3$.
We often depict these with energy diagrams including a ``virtual level'' which is non-physical, but show the conservation of energy.\\
\includegraphics*[width=12cm]{4-30-1} \\
We start by making the monochromatic approximation, where $\partial_t A_l = 0$, considering the case of SFG:
\begin{align*}
	P_3^{NL}(z) &= \epsilon_0 \chi^{(2)}A_1A_2 e^{i(k_1 + k_2)z} \\
	P_2^{NL} &= \epsilon_0 \chi^{(2)} A_3 A_1^* e^{i(k_3-k_1)z}
\end{align*}
So then:
\begin{align*}
	\partial_z A_3 &= \frac{i\omega_3}{2\epsilon_0 cn_3} e^{-ik_3z} \epsilon_0 \chi^{(2)} A_1A_2 e^{i(k_1 + k_2)z} \\
	&= \frac{i\omega_3\chi^{(2)}}{2cn_3} e^{-ik_3z} e^{i\Delta k z}A_1A_2  & \Delta k &= k_1 + k_2 - k_3 \\
	\partial_z A_2 &= \frac{i\omega_2\chi^{(2)}}{2cn_2} e^{-ik_3z} e^{-i\Delta k z}A_3A_1^* \\
	\partial_z A_1 &= \frac{i\omega_1\chi^{(2)}}{2cn_1} e^{-ik_3z} e^{-i\Delta k z}A_3A_2^*
\end{align*}
If we pick the special case of second harmonic, where $\omega_1=\omega_2=\frac{1}{2}\omega_3$, then:
\begin{align*}
	\partial_z A_1 &= \frac{i\omega_1\chi^{(2)}}{2cn_1} e^{-i\Delta kz} A_3 A_1^* \\
	\partial_z A_3 &= \frac{i\omega_3\chi^{(2)}}{2cn_3} e^{i\Delta kz} A_1^2
\end{align*}
We can solve this in a couple ways. We first concider the ``undepleted pump'' approximation. In this approximation $A_1$ is constant. Therefore:
\begin{align*}
	\partial_z A_3 &= \frac{i\omega_3\chi^{(2)}}{2cn_3} e^{i\Delta kz} A_1^2 \\
	A_3 &= \frac{i\omega_3\chi^{(2)}}{2cn_3} A_1^2 \frac{e^{i\Delta kL} - 1}{i\Delta k} \\
	&= \frac{i\omega_3\chi^{(2)}}{2cn_3} A_1^2 2 e^{i\frac{\Delta k L}{2}} L\text{sinc} \frac{\Delta k L}{2}
\end{align*}
Where $\text{sinc}\frac{\Delta k L}{2} e^{i \frac{\Delta k L}{2}}$ is called the phase matching function.
If $\Delta k=0$ we say we are phase matched. In order to have this in our current process, we need $n_1 = n_3$. If we are not phase matched we will not effecient generation.
In a normal material we have a monotonically increasing index of refraction as we change frequency. In order to get phase matching we typically take advantage of birefringence.
We can have type I in which $\omega_1$ and $\omega_2$ along the ordinary axis, and $\omega_3$ is along the extraordinary. Type II will have $\omega_1$ ordinary and $\omega_2$ and $\omega_3$ along the extraordinary.

In order to model this we look to the Sellemeir equations, which describe our indicies of refraction in terms of temperature and wavelength.

If we now want to look at generating things with a finite bandwidth:
\begin{align*}
	\Delta k &= 2k_1 -k_3 & \omega_1 &= \omega_{10} \pm \nu/2 & \omega_3 &= \omega_{30} \pm \nu \\
	\Delta k &\approx 2(k_{10} \pm k_{10}' \nu/2) - k_{30} \mp k_{30}'\nu \\
	&\approx k_{10}' \nu - k_{30}'\nu \\
	&\approx \left(\frac{1}{v_1} - \frac{1}{v_3}\right)\nu
\end{align*}
So:
\begin{align*}
	|A_3|^2 &= \Big|\frac{\omega_3\chi^{(2)}}{2cn_3}\Big| I_1^2 \text{sinc}^2\left(\frac{L}{2}\left(\frac{1}{v_1} - \frac{1}{v_3}\right)\nu\right)
\end{align*}
We define $\tau = \frac{L}{v_l}$ which is the time it takes a pulse at frequency $\omega_l$ to propogate through our material, so:
\begin{align*}
	|A_3|^2 &= \Big|\frac{\omega_3\chi^{(2)}}{2cn_3}\Big| I_1^2 \text{sinc}^2\frac{\nu (\tau_1-\tau_3)}{2}
\end{align*}
So our bandwidth is:
\begin{align*}
	\Delta\nu &= \frac{2\pi}{\tau_1-\tau_3}
\end{align*}
If we now return to the more general case of Sum Frequency Generation:
\begin{align*}
	\partial_z A_3 &= \frac{i\omega_3}{2\epsilon_0 cn_3} e^{-ik_3 z} \epsilon_0 \chi^{(2)} 2 A_1 A_2 e^{i(k_1 + k_2)z} \\
	\partial_z A_2 &= \frac{i\omega_2}{2\epsilon_0 cn_2} \epsilon_0 \chi^{(2)} 2 A_3 A_1^* e^{-i\Delta k z} \\
	\partial_z A_3 &= \frac{i\omega_1}{2\epsilon_0 cn_1} \epsilon_0 \chi^{(2)} 2 A_3 A_2^* e^{-i\Delta k z}
\end{align*}
