We can find our eigenstates for $\hat{H}_p$:
\begin{align*}
	\hat{H}_p \ket{E} &= E\ket{E}
\end{align*}
Which will be our standard solutions for Hydrogen/Helium/etc. We call these the ``bare'' states of the material system.

Our radiation field will have solutions:
\begin{align*}
	\hat{H}_R &= \sum_l \hbar\omega_l \left(\hat{a}_l^\dagger\hat{a}+l +\frac{1}{2}\right) \\
	\hat{a}^\dagger_l\hat{a}\ket{n_l} &= n_l\ket{n_l} \\
	\hat{H}_R\ket{\{n_l\}} &= \sum_l \hbar\omega_l\left(n_l + \frac{1}{2}\right)
\end{align*}
These are the ``bare'' states of the electrommagnetic field.
We will now treat our interaction as a perturbation of the bare states of the field and the matter:
\begin{align*}
	\hat{E} \otimes \ket{\{n_l\}}
\end{align*}
In order to quantize our interaction we say:
\begin{align*}
	\hat{\bm{E}}_\perp(\bm{x},t) &= \sum_l \mathcal{E}_l \bm{u}_l(\bm{x},t) \hat{a}_l + \hercon
\end{align*}
We will now focus on one ``atom'' (or superconducting qubit, molecule, etc.). We say:
\begin{align*}
	\hat{H}_p &= \sum_i E_i \ket{i}\bra{i} \\
	&= \sum_i E_i \hat{\sigma}_{ii} \\
	\hat{\sigma}_{ij} &= \ket{i}\bra{j} \\
	\hat{\bm{d}} &= e\hat{\bm{x}} \\
	&= e\sum_{ij} \ket{i}\bra{i}\hat{\bm{x}} \ket{j}\bra{j} \\
	&= \sum_{ij} \bm{d}_{ij}\hat{\sigma}_{ij} \\
	\bm{d}_{ij} &= e\bra{i}\hat{\bm{x}}\ket{j}
\end{align*}
We choose to work in terms of plane wave modes (dropping the distinction for our perpendicular modes):
\begin{align*}
	\bm{E}(\bm{x},t) &= i\sum_{\bm{k},l} \bm{e}_{\bm{k},l}\mathcal{E}_{\bm{k}}\left(\hat{a}_{\bm{k},l} e^{i(\bm{k}\cdot\bm{x} -\omega_kt)}  + \hercon\right) \\
	\hat{H}_I &= \sum_{ij}\sum_{\bm{k},l} \hbar\left(\frac{-i\bm{d}_{ij}\cdot\bm{e}_{\bm{k},l} \mathcal{E}_{\bm{k}}}{\hbar}\right) \hat{\sigma}_{ij}\left[\hat{a}_{\bm{k},l} e^{i(\bm{k}\cdot\bm{x} -\omega_k t)} +\hercon\right) \\
	\Omega_{\bm{k},l}^{ij} &= \frac{-i\bm{d}_{ij}\cdot\bm{e}_{\bm{k},l}\mathcal{E}_{\bm{k}}}{\hbar}
\end{align*}
We note that the lifetime of atomic states here will be related to vacuum fluctuations that appear in second order perturbation theory.

\subsection{Jaynes Cummings Model}
We now consider only two level atomic systems that interact with one mode of the electrommagnetic field.
We further simplify this by setting zero energy centrally, between the ground and excited energies. Therfore our energies are:
\begin{align*}
	E_g &= -\frac{\hbar\omega_{eg}}{2} & E_e &= \frac{\hbar\omega_{eg}}{2}
\end{align*}
So:
\begin{align*}
	\hat{H}_p &= \frac{\hbar\omega_{eg}}{2}\hat{\sigma}_3
\end{align*}
And we introduce raising and lowering operators:
\begin{align*}
	\hat{\sigma}_+ &= \ket{e}\bra{g} & \hat{\sigma}_- &= \ket{g}\bra{e}
\end{align*}
And then we can write:
\begin{align*}
	\hat{H}_I &= \hbar\Omega(\hat{\sigma}_+ +\hat{\sigma}_-)(\hat{a}_\omega e^{-i\omega t} + \hat{a}_\omega^\dagger e^{i\omega t}) \\
	\Omega &= \frac{-i\bm{d}_{eg} \bm{e}_k \mathcal{E}_k}{\hbar} & \mathcal{E}_k &= \sqrt{\frac{\hbar\omega_k}{2\epsilon_0 V}}
\end{align*}
So then:
\begin{align*}
	\hat{H}_{J.C.} &= \frac{\hbar\omega_{eg}}{2} \hat{\sigma}_3 + \hbar\omega\hat{a}_\omega^\dagger\hat{a}_\omega + \hbar\Omega(\hat{\sigma}_+ + \hat{\sigma}_-)(\hat{a}_\omega e^{-i\omega t} + \hat{a}_\omega^\dagger e^{i\omega t})
\end{align*}
If we time evolve our system without including our perturbations we can say our raising and lowering operators evolve as:
\begin{align*}
	\hat{\sigma}_\pm(t) &= \hat{\sigma}(0) e^{\pm i\omega_{eg} t}
\end{align*}
So then:
\begin{align*}
	\hat{H}_{J.C.} &= \frac{\hbar\omega_{eg}}{2} \hat{\sigma}_3 + \hbar\omega\hat{a}_\omega^\dagger\hat{a}_\omega + \hbar\Omega\left(\hat{\sigma}_- \hat{a}e^{-i(\omega + \omega_{eg})t} + \hat{\sigma}_+\hat{a} e^{-i(\omega - \omega_{eg})t}
		\hat{\sigma}_-\hat{a}^\dagger e^{i(\omega - \omega_{eg})t} + \hat{\sigma}_+\hat{a}^\dagger e^{i(\omega + \omega_{eg})t}\right)
\end{align*}
Taking the rotating wave approximation (which physically here kills off terms that don't conserve energy):
\begin{align*}
	\hat{H}_{J.C.} &= \frac{\hbar\omega_{eg}}{2} \hat{\sigma}_3 + \hbar\omega\hat{a}_\omega^\dagger\hat{a}_\omega + \hbar\Omega\left(\hat{\sigma}_+\hat{a} e^{-i(\omega - \omega_{eg})t} + \hat{\sigma}_-\hat{a}^\dagger e^{i(\omega - \omega_{eg})t}\right) \\
	\hat{H}_{J.C.} &= \frac{\hbar\omega_{eg}}{2} \hat{\sigma}_3 + \hbar\omega\hat{a}_\omega^\dagger\hat{a}_\omega + \hbar\Omega\left(\hat{\sigma}_+\hat{a} e^{i\Delta t} + \hat{\sigma}_-\hat{a}^\dagger e^{-i\Delta t}\right) \\
	\Delta &= \omega_{eg} - \omega
\end{align*}
We note that $\hat{P}_e = \ket{e}\bra{e} + \ket{g}\bra{g}$ is conserved (which is just saying we have one atom), as well as $\hat{N}_{ex} = \ket{e}\bra{e} + \hat{a}^\dagger \hat{a}$ (which says the number of excitations is conserved).
Using this we break up our Hamiltonian into:
\begin{align*}
	\hat{H}_1 &= \hbar\omega\hat{N}_{ex} + \hbar\left(\frac{\omega_{eg}}{2} - \omega\right)\hat{P}_e \\
	\hat{H}_2(0) &= -\hbar\Delta\ket{g}\bra{g} + \hbar\Omega(\hat{\sigma}_+\hat{a} + \hat{\sigma}_-\hat{a}^\dagger) \\
	[\hat{H}_1,\hat{H}_2] &= 0
\end{align*}
While $\hat{H}_1$ will only lead to phase variations, $\hat{H}_2$ will cause dynamics.

We now consider the example of an on resonance ($\Delta=0$) and initial state $\ket{\psi(0)} = \ket{e}\ket{n}$. We now want to determine $\ket{\psi(t)}$:
\begin{align*}
	\ket{\psi(t)} &= c_g(t) \ket{g}\ket{n+1} + c_e(t)\ket{e}\ket{n}
\end{align*}
We move to the interaction picture where:
\begin{align*}
	i\hbar\partial_t\ket{\psi(t)}_I &= \hat{\tilde{H}}_I\ket{\psi(t)}_I \\
	\hat{\tilde{H}}_I &= -\hbar\Delta\ket{g}\bra{g} + \hbar\Omega(\hat{\sigma}_+\hat{a} + \hat{\sigma}_-\hat{a}^\dagger)
\end{align*}
So our Schroedinger equation becomes:
\begin{align*}
	i\hbar \dot{\tilde{c}}_g\ket{g}\ket{n+1} + i\hbar\dot{\tilde{c}}_e\ket{e}\ket{n} &= \hbar\Omega\left[\tilde{c}_g\sqrt{n+1}\ket{e}\ket{n} + \tilde{c}_e\sqrt{n+1}\ket{g}\ket{n+1}\right] \\
	i\hbar\dot{\tilde{c}}_g &= \hbar\Omega\sqrt{n+1}\tilde{c}_e & i\hbar\dot{\tilde{c}}_e &= \hbar\Omega\sqrt{n+1}\tilde{c}_g
\end{align*}
Which we can solve by looking at second dervitives, so:
\begin{align*}
	\ddot{\tilde{c}}_g &= -\Omega^2(n+1)\tilde{c}_g \\
	\tilde{c}_g(t) &= \sin(\Omega\sqrt{n+1}t)
\end{align*}
For on resonance the energy of both our states are identical so there is no additional phase imposed by the interaction picture so we can say:
\begin{align*}
	\ket{\psi(t)} &= -i\sin(\Omega\sqrt{n+1}t)\ket{g}\ket{n+1} +\cos(\Omega\sqrt{n+1}t)\ket{e}\ket{n}
\end{align*}
This experiences Rabi oscilations, even if we don't include any photons in our initial state.

We now look at the initil state $\ket{\psi(0)} = (c_e(0)\ket{e} + c_g(0)\ket{g})(\sum_n c_n\ket{n})$ which turns out to only have nearest neighbor interactions between the modes.
