\subsection{Single photon states}
We start by considering a signle photon that may be distributed across multiple modes. 

If we begin by considering a photon in a single mode (lets say mode $3$) we can write this state as:
\begin{align*}
	\ket{\psi} &= \hat{a}_3^\dagger\vac \\
	\ket{\psi} &= \ket{1}_3
\end{align*}
Here all the measurable data you can find for a single photon depends on the mode. For solid state single photon emitters you will see that the emitter will not emit into the same mode each time.
This will be an inghomogeneous broadening to our system. This can be described by a single photon mixed state/mixed mode:
\begin{align*}
	\rho &= \sum_{i,j} p_{ij}\ket{1}_i\bra{1}_j
\end{align*}
So far we've been using plane wave modes to represent the state of our system. For our lacalized/localizable system we would rather work with wave packet modes. These wave packet modes are non-monochromatic, and we can write:
\begin{align*}
	\bm{v}_m(\bm{x}) &= \sum_l U_{ml} \bm{u}_l(\bm{x}
\end{align*}
We want our transformation $U$ to be unitary in order to preserve orthogonality, magnitude and inner products. In our original system, our inner product was defined:
\begin{align*}
	\bm{a}\cdot\bm{b} &= \frac{1}{V} \int d^3 x \bm{a}^*\cdot\bm{b}
\end{align*}
We can express our old basis in terms of the new basis:
\begin{align*}
	\bm{u}_l  &= \sum_m U_{lm}^* \bm{v}_m
\end{align*}
So:
\begin{align*}
	\hat{\bm{E}} &= \sum_l i \mathcal{E}_l \sum_m U_{lm}^*\bm{v}_m e^{i\omega_l r} \hat{a}_l + \hercon \\
	\hat{\bm{E}} &= \sum_m \bm{v}_m \sum_l i \mathcal{E}_l U_{lm}^* e^{i\omega_l r}\hat{a}_l + \hercon
\end{align*}
If we choose monochromatic waves then:
\begin{align*}
	\hat{\bm{E}} &= \sum_m \bm{v}_m\mathcal{E}_mi e^{i\omega_m r}\hat{b}_m + \hercon &
	\hat{b}_m &= \sum_l U_{lm}^* \hat{a}_l
\end{align*}
We can see:
\begin{align*}
	[\hat{b}_m,\hat{b}_{m'}^\dagger] &= \sum_{l,l'} [U_{lm}^*\hat{a}_l,U_{m'l'}\hat{a}_{l'}^\dagger] \\
	[\hat{b}_m,\hat{b}_{m'}^\dagger] &= \sum_{l,l'} U_{lm}^*U_{m'l'}[\hat{a}_l,\hat{a}_{l'}^\dagger] \\
	[\hat{b}_m,\hat{b}_{m'}^\dagger] &= \sum_{l,l'} U_{lm}^*U_{m'l'}\delta_{ll'} \\
	[\hat{b}_m,\hat{b}_{m'}^\dagger] &= \sum_{l,l'} U_{lm}^*U_{m'l'}\delta_{ll'} \\
	[\hat{b}_m,\hat{b}_{m'}^\dagger] &= \sum_{l,l'} U_{lm}^*U_{m'l} \\
	[\hat{b}_m,\hat{b}_{m'}^\dagger] &= \delta_{mm'}
\end{align*}
So our unitary mode transformations preserve bosonic commutators.

We now look at pulse modes, which are non-monochromatic. Sometimes these are called temporal mmodes. For an atom we can express our spectral envelope:
\begin{align*}
	\tilde{A}(\omega) &\propto \frac{1}{\frac{\Gamma^2}{4} + (\omega - \omega_0)^2} \\
	A(t) &= a(t)  e^{-i\omega_0 t} \\
	a(t) &\propto e^{-\frac{t}{\tau}}
\end{align*}
These Lorentzian's are one of the modes we will commonly expand about, we will also commonly use the Hermite Gaussian pulses:
\begin{align*}
	v_m(t) &\propto HG_m(t) e{-\frac{t^2}{\tau^2}}
\end{align*}
We now rewrite $E$:
\begin{align*}
	\hat{\bm{E}} &= i\sum_k \sqrt{\frac{\hbar\omega_k}{2\epsilon_0 V}} \hat{x} e^{i(kz-\omega_k t} \hat{a}_k - \hercon \\
	v_m &= \sum_k U_{mk} e^{ikz} \\
	\psi_m(\tau) &= \sum_\omega \tilde{\psi}_m(\omega) e^{-i\omega\tau} & \tau &= t- \frac{z}{c}
\end{align*}
This yields:
\begin{align*}
	\hat{\bm{E}} &= i\sum_\omega \sqrt{\frac{\hbar}{2\epsilon_0 V}}\sqrt{\omega} \hat{x} e^{-i\omega\tau} \hat{a}_\omega - \hercon \\
	e^{-i\omega\tau} &= \sum_m \tilde{\psi}_{m}^*(\omega) \psi_m(\tau) \\
	\hat{\bm{E}} &= i\sum_\omega \sum_m \sqrt{\frac{\hbar}{2\epsilon_0 V}}\sqrt{\omega} \hat{x} \tilde{\psi}_{m}^*(\omega) \psi_m(\tau) \hat{a}_\omega - \hercon \\
	\hat{\bm{E}} &= i\sqrt{\frac{\hbar}{2\epsilon_0 V}} \hat{x}\sum_m  \psi_m(\tau) \sum_\omega \sqrt{\omega}\tilde{\psi}_{m}^*(\omega) \hat{a}_\omega - \hercon
\end{align*}
If $tilde{\psi}_m^*(\omega)$ is nearly constant around some $\omega_0$ we can say:
\begin{align*}
	\hat{b}_m &= \sum_\omega \sqrt{\omega} \tilde{\psi}_m^*(\omega) \\
	\hat{b}_m &= \sqrt{\omega_0} \sum_\omega \tilde{\psi}_m^*(\omega) \\
	[\hat{b}_m, \hat{b}_{m'}^\dagger] &= \omega_0\delta_{mm'}
\end{align*}
Taking the continuum limit we see:
\begin{align*}
	\sum_\omega &\to \int \frac{d\omega}{2\pi} \\
	\sum_k &\to \frac{L}{2\pi} \int dk \\
	f(t) &= \int \frac{d\omega}{2\pi} e^{-i\omega t}\tilde{f}(\omega) \\
	\tilde{f}(\omega) &= \int dt e^{i\omega t}f(t) \\
	\hat{b}_m &= \int \frac{d\omega}{2\pi} \sqrt{\omega} \tilde{\psi}_m^*(\omega \hat{a}(\omega) \\
	[\hat{a}(\omega),\hat{a}^\dagger(\omega')] &= 2\pi \delta(\omega - \omega') \\
	[\hat{a}(t),\hat{a}^\dagger(t') &= \delta(t-t')
\end{align*}
We can see:
\begin{align*}
	e^{-i\omega\tau} &= \sum_m \psi_{m\omega}^* \psi_m(\tau) \\
	\psi_m(\tau) &= \sum_{\omega'} \psi_{m\omega'} e^{-i\omega'\tau} \\
	e^{-i\omega\tau} &= \sum_m \psi_{m\omega}^* \sum_{\omega'} \psi_{m\omega'} e^{-i\omega'\tau} \\
	e^{-i\omega\tau} &= \sum_{\omega'}\delta_{\omega\omega'} e^{-i\omega'\tau} \\
	e^{-i\omega\tau} &= e^{-i\omega\tau}
\end{align*}
Which is what we expect. 

We now consider the commutator between two different non orthogonal modes:
\begin{align*}
	[\hat{a}_\psi, \hat{a}^\dagger_\phi] &= (\psi|\phi)
\end{align*}
Where $(\psi|\phi)$ is the overlap between the two modes.
\subsection{Two photon states}
We look at a simple two photon state:
\begin{align*}
	\frac{(\hat{a}_3^\dagger)^2}{\sqrt{2}}\vac &= \ket{2}_3
\end{align*}
Or:
\begin{align*}
	\hat{a}_3^\dagger \hat{a}_4^\dagger \vac &= \ket{1}_3\ket{1}_4
\end{align*}
But we can have a more complicated state:
\begin{align*}
	\hat{a}_\psi^\dagger\hat{a}_\phi^\dagger\vac
\end{align*}
\subsection{Single mode states}
We say for a pure single mode state:
\begin{align*}
	\ket{\psi} &= \sum_n c_n \ket{n}
\end{align*}
Where we would describe this as a state of the photon number representation.

We could instead use the quadrature representation:
\begin{align*}
	\bra{q}\ket{p} &= e^{-iqp} \\
	\bra{q}\ket{\psi} &= \psi(q) \\
	\bra{p}\ket{\psi} &= \tilde{\psi}(p)
\end{align*}
Where are $q$ and $p$ states are complete:
\begin{align*}
	1 &= \int dq \ket{q}\bra{q} \\
	1 &= \int \frac{dp}{2\pi} \ket{p}\bra{p}
\end{align*}
And we can then write:
\begin{align*}
	\ket{\psi} &= \int \psi(q) \ket{q} dq \\
	\ket{\psi} &= \int \frac{dp}{2\pi} \tilde{\psi}(p) \ket{p}
\end{align*}
Where we can say $|c_n|^2$ represents the chance of finding $n$ photons, $|\psi|^2 dq$ represents the chance of finding a photon near x, and $|\tilde{\psi}|^2 dp$ is the chance of finding a photon with $p$
