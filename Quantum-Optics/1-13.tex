We can define a unit vector:
\begin{align*}
	\bm{u}_l(\bm{x} &= \bm{\epsilon}_l e^{i\bm{k}_l\cdot\bm{x}}
\end{align*}
Which we call our ``plane wave mode''. We know:
\begin{align*}
	\del\times\bm{u}_l(\bm{x} &= i\bm{k}_l\times\bm{\epsilon}_l e^{i\bm{k}\cdot\bm{x}}
\end{align*}
These modes are orthonormal by construction:
\begin{align*}
	\frac{1}{V} \int d^3 x\bm{u}_l^*(\bm{x})\cdot\bm{u}_m(\bm{x}) &= \delta_{lm} \\
	\frac{1}{V} \int d^3 x(\del\times\bm{u}_l^*(\bm{x}))\cdot(\del\times\bm{u}_m(\bm{x})) &= \delta_{lm}k_l^2
\end{align*}
We now have constructed a set of modes, which are solutions to Maxwell's equations. Since these are complete, and Maxwell's equations are linear we can write our solutions in terms of sums of these modes.
Additionally we can choose a different set of modes, and this will be related to our current set of modes via a unitary transformation:
\begin{align*}
	v_l(\bm{x},t) &= U_{lm} \bm{u}_m
\end{align*}
We can write our fields in terms of our modes:
\begin{align*}
	\bm{E}(\bm{x},t) &= \sum_l i\omega_l A_l \bm{u}_l(\bm{x},t)+\compcon \\
	\bm{B}(\bm{x},t) &= \sum_l i\bm{k}_l\times  \bm{u}_l(\bm{x},t)A_l+\compcon \\
	\bm{E}(\bm{x},t) &= \sum_l A_l \bm{u}_l(\bm{x},t)+\compcon
\end{align*}
We now seek to write our Hamiltonian for this system:
\begin{align*}
	H_R &= \frac{\epsilon_0}{2}\int d^3 x (|E|^2 + c^2|B|^2) \\
	H_R &= \frac{\epsilon_0}{2}\int d^3 x \left( \sum_l\sum_{l'} (i\omega_lA_l\bm{u}_l - i\omega_lA_l^* \bm{u}_l^*)\cdot(-i\omega_{l'}A_{l'}^* \bm{u}_{l'}^* + i\omega_{l'} A_{l'}\bm{u}_{l'}) + |B|^2\right)
\end{align*}
And we know that in our integral we can say:
\begin{align*}
	\bm{u}_l\cdot\bm{u}_{l'}^* \to \delta_{l,l'} \\
	\bm{u}_l^*\cdot\bm{u}_{l'} \to \delta_{l,l'} \\
	\bm{u}_l\cdot\bm{u}_{l'} \to \delta_{l,-l'} \\
	\bm{u}_l^*\cdot\bm{u}_{l'}^* \to \delta_{l,-l'}
\end{align*}
Once we complete the algebra we find:
\begin{align*}
	H_R &= \frac{\epsilon_0}{2} \sum_{l,l'} \left[(i\omega_l)(-i\omega_{l'})\delta_{l,l'}V A_l A_{l'}^* + A_l A_{l'} (i\omega_l)(i\omega_{l'})V\delta_{l,-l'} +
			(-i\omega_l)(i\omega_{l'}) V\delta_{l,l'} A_l^*A_{l'} + A_l^* A_{l'}^* (i\omega_l)(i\omega_{l'}) V\delta_{l,-l'} \right] \\
	    &+ \frac{\epsilon_0c^2}{2}\sum_{l,l'} \left[(i\bm{k}_l)\cdot(-i\bm{k}_{l'}) V\delta_{l,l'} A_lA_{l'}^* + (i\bm{k}_l \times\bm{\epsilon}_l)\cdot(i\bm{k}_{l'}\times\bm{\epsilon}_{l'}) A_l A_{l'} \int e^{i(\bm{k}_l + \bm{k}_{l'})\cdot x} 
	    +(-i\bm{k}_l)\cdot(i\bm{k}_{l'}) V\delta_{l,l'} A_l^*A_{l'} + (-i\bm{k}_l \times\bm{\epsilon}_l)\cdot(-i\bm{k}_{l'}\times\bm{\epsilon}_{l'}) A_l^* A_{l'}^* \int e^{-i(\bm{k}_l + \bm{k}_{l'})\cdot x} \right]
\end{align*}
The terms involving $\delta_{l,-l}$ will all cancel, while all the other terms are identical, so we are then left with:
\begin{align*}
	H_R &= 2\epsilon_0 \sum_l \omega_l^2V |A_l|^2 
\end{align*}
Where then:
\begin{align*}
	A_l &= \frac{1}{V} \int d^3 x \bm{u}_l^* (\bm{x},t) \cdot\bm{E}(\bm{x},t)
\end{align*}
Alternatively we could write this as:
\begin{align*}
	H_R &= 2\epsilon_0 \sum_l V |E_l|^2 
\end{align*}
We now introduce the terms:
\begin{align*}
	Q_l &= \sqrt{V\epsilon_0\omega_l}(A_l + A_l^*) \\
	P_l &= \frac{1}{i}\sqrt{V\epsilon_0\omega_l}(A_l - A_l^*) \\
	A_l &= \frac{Q_l + iP_l}{2\sqrt{V\epsilon_0\omega_l}} \\
	A_l^* &= \frac{Q_l - iP_l}{2\sqrt{V\epsilon_0\omega_l}}
\end{align*}
So then:
\begin{align*}
	H_R &= 2\epsilon_0\sum_l V\omega_l^2\frac{Q_l^2 + P_l^2}{4V\epsilon_0\omega_l} \\
	H_R &= \frac{1}{2}\sum_l \omega_l(Q_l^2 + P_l^2)
\end{align*}
It turns out that these two are canonically conjugate variables:
\begin{align*}
	\partder{H}{P_l} &= \omega_l P_l &
	\partder{H}{Q_l} &= \omega_l Q_l \\
	\dot{Q}_l &= \omega_l P_l &
	\dot{P}_l &= -\omega_l Q_l
\end{align*}
\subsection{Canonical quaantization of the EM field}
We now have our electric field writen as a sum of Harmonic oscilators. We have identified our canonically conjugate variable ($Q_l,P_l$), and now we impose our commutation relations on the corresponding operators:
\begin{align*}
	[\hat{Q}_l,\hat{P}_{l'}] &= i\hbar\delta_{l,l'}
\end{align*}
We hace our operator for $A$ now:
\begin{align*}
	\hat{A}_l &= \frac{1}{\sqrt{4 V \epsilon_0\omega_l}} (\hat{Q}_l + i\hat{P}_l) \\
	\hat{A}_l^\dagger &= \frac{1}{\sqrt{4 V \epsilon_0\omega_l}} (\hat{Q}_l - i\hat{P}_l)
\end{align*}
We now define the scaled operators:
\begin{align*}
	\hat{a}_l &= \sqrt{\frac{2V\epsilon_0\omega_l}{\hbar}} A_l \\
	\hat{a}_l^\dagger &= \sqrt{\frac{2V\epsilon_0\omega_l}{\hbar}} A_l^\dagger \\
	[\hat{A}_l,\hat{A}_{l'}^\dagger] &= \frac{1}{4V\epsilon_0\omega_l}\left(-i[\hat{Q}_l,\hat{P}_{l'}] + i[\hat{P}_l,\hat{Q}_{l'}]\right) \\
	[\hat{A}_l,\hat{A}_{l'}^\dagger] &= \frac{i}{2V\epsilon_0\omega_l}\delta_{l,l'}
	[\hat{a}_l,\hat{A}_{l'}^\dagger] &= \delta_{l,l'}
\end{align*}
Where these $\hat{a}$ operators are our creation and annihilation operators for a specific mode $l$. These modes are ``vessels'' that contain excitions. These excitations are what we refer to as photons.

We now rewrite our conjugate coordinates in terms of the creation and anihilation operators:
\begin{align*}
	\hat{Q}_l &= \sqrt{2\hbar} (\hat{a} + \hat{a}^\dagger) \\
	\hat{P}_l &= -i\sqrt{2\hbar} (\hat{a} - \hat{a}^\dagger)
\end{align*}
So our Hamiltonian becomes:
\begin{align*}
	\hat{H}_R &= \sum_l \frac{\hbar\omega_l}{2}(\hat{a}\hat{a}^\dagger + \hat{a}^\dagger\hat{a}) \\
	\hat{H}_R &= \sum_l \frac{\hbar\omega_l}{2}(2\hat{a}^\dagger\hat{a}^\dagger + 1)
\end{align*}
