\chapter{Winter Term, Brian Smith}
\section{Review}
\subsection{Classical Mechanics}
We can look at classical mechanics in terms of three different formalizims, all of these describe how classical particles move in space.

The first is the Newtonian model in which we describe things in terms of forces applied, i.e. $\sum\bm{F}  = m\ddot{\bm{x}}$.
As an example, take a particle moving with mass m in 1 dimension with an arbitrary potential, so:
\begin{align*}
	F &= -\partder{V}{x} \\
	m\ddot{x} &= -\partder{V}{x}
\end{align*}

The second is the Lagrangian formulation in which $L = T -V$, the equations of motion determined from $\delta S = 0$ where $S = \int L dt$. In other words this turns our problem into a variational calculus  problem.
\begin{align*}
	\delta S  &= \int \left(\partder{L}{q} \delta q + \partder{L}{\dot{q}} \delta \dot{q}\right) dt \\
	\delta S  &= \int \left(\partder{L}{q} \delta q - \frac{d}{dt}\partder{L}{\dot{q}} \delta q\right) dt \\
	\delta S  &= \int \left(\partder{L}{q}  - \frac{d}{dt}\partder{L}{\dot{q}}\right)\delta q dt \\
	\partial_q L &= \frac{d}{dt}\partder{L}{\dot{q}}
\end{align*}

As an example, take a particle moving with mass m in 1 dimension with an arbitrary potential, so:
\begin{align*}
	L &= \frac{m}{2}\dot{x}^2 - V(x) \\
	\partder{L}{x} &= -\partder{V}{x} \\
	\partder{L}{{\dot{x}}} &= m \dot{x} & \text{We call this term the conjugate momentum } p_x\\
	m \ddot{x} &= -\partder{V}{x}
\end{align*}
So here we see the result matches what we saw for Newtonian mechanics.

The third is the Hamiltonian formulation in which $H = \sum p_i \dot{q}_i - L$ (the legendre transform of the Lagrangian formulations). Here we first determine our conjugate momenta $p_j = \partder{L}{\dot{q}_j}$.
Then we construct $H = p_j \dot{q}_j - L$. Now we minimize the action again in terms of $q_j$ and $p_j$, but that is unweildy, so:
\begin{align*}
	\partder{H}{q} &= -\partder{q}{L} - \partder{L}{\dot{q}} \partder{\dot{q}}{q} + \partder{}{q} (p \dot{q}) \\
	\partder{H}{q} &= -\partder{q}{L} - \partder{L}{\dot{q}} \partder{\dot{q}}{q} + \partder{p}{q} \dot{q} + \partder{\dot{q}}{q} p \\
	\partder{H}{q} &= -\partder{q}{L} - \partder{L}{\dot{q}} \partder{\dot{q}}{q} + \partder{\dot{q}}{q} p \\
	\partder{H}{q} &= -\partder{L}{q} \\
	\partder{H}{q} &= -\dot{p}
\end{align*}
We now look at our other equation of motion:
\begin{align*}
	\partder{H}{p} &= \partder{}{p} (p\dot{q}) - \partder{L}{p} \\
	\partder{H}{p} &= \dot{q}
\end{align*}
Looking again at a particle with mass m in 1 dimension with an arbitrary potential:
\begin{align*}
	p &= m\dot{x} \\
	H &= p\dot{x} - L \\
	H &= \frac{p^2}{m} - \frac{p^2}{2m} + V(x) \\ 
	H &= \frac{p^2}{2m}  + V(x) \\
	\dot{x} &= \frac{p}{m} \\
	-\dot{p} &= \partder{V}{x} \\
	\ddot{x} &= \frac{\dot{p}}{m} \\
	m\ddot{x} &= -\partder{V}{x}
\end{align*}
Which again matches the Newtonian approach.

Alternatively we could use the Poisson formalism. We say the Poisson bracket is defined:
\begin{align*}
	\{f,g\} &= \sum_i \partder{f}{q_i}\partder{g}{p_i} - \partder{g}{q_i}\partder{f}{p_i}
\end{align*}

We can see that for any quantity:
\begin{align*}
	\{f,H\} &= \dot{f}
\end{align*}

If we now look at the motion of a pendulum:\\ 
\includegraphics*[width=8cm]{images/1-06-fig1.png} \\
We can quickly derive that:
\begin{align*}
	q &= l\theta \\
	V &= mgy \\
	p &= m\dot{\theta} \\
	\omega &= \frac{g}{q} \\
	H &= \frac{1}{2m} p^2 + \frac{m}{2} \omega q^2
\end{align*}

If we want to quantize a system we do two things: \\
0) Determine $H$, $p$ and $q$ for our classical system, such that the Hamilton equations of motion give correct classical dynamics. These must be canonically conjugate variables such that $p_i = \partder{L}{\dot{q}_i}$. \\
1) We then determine the Poisson brackets for $q_i$ and $p_i$. \\
2) We then change our canonically conjugate variables to operators acting on a Hilbert space, with commutators given by $[\hat{a},\hat{b}] = i\hbar\{a,b\}$

Now to quantize our pendulum:
\begin{align*}
	\hat{H} &= \frac{\hat{p}^2}{2m} + \frac{m\omega^2}{2}\hat{q}^2
\end{align*}
To solve this we use ladder/creation/raising/anihilation/lowering operators:
\begin{align*}
	\hat{a}^\dagger &= \sqrt{\frac{mw}{2\hbar}} \left(\hat{q} - \frac{i}{m\omega} \hat{p}\right) \\
	\hat{a} &= \sqrt{\frac{mw}{2\hbar}} \left(\hat{q} + \frac{i}{m\omega} \hat{p}\right)
\end{align*}
These operators are non-Hermition, and therefore they can't represent an observable. If we say we have energy eigenstates $\hat{H}\ket{\psi} = E\ket{\psi}$, we claim that $\hat{H}\hat{a}^\dagger \ket{\psi} = (E + \hbar\omega)\ket{\psi}$.
In order to show this we rewrite our Hamiltonian:
\begin{align*}
	\hat{q} &= \sqrt{\frac{2\hbar}{m\omega}} \frac{\hat{a} + \hat{a}^\dagger}{2} \\
	\hat{p} &= \sqrt{\frac{2\hbar}{m\omega}} m\omega\frac{\hat{a} - \hat{a}^\dagger}{2i}
\end{align*}
