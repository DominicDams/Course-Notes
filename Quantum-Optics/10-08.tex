\subsection{Unitary Transformations}
We change basis by using the unitary transform:
\begin{align*}
	\ket{\psi'} &= U\ket{\psi}
\end{align*}
And we want our new Hamiltonian $H'$ s.t.:
\begin{align*}
	i\hbar \partial_t \ket{\psi'} &= H'\ket{\psi'}
\end{align*}
So starting from our untransformed term:
\begin{align*}
	i\hbar \partial_t \ket{\psi} &= H\ket{\psi} \\
	i\hbar \partial_t U^\dagger\ket{\psi'} &= HU^\dagger\ket{\psi'}\\
	Ui\hbar \partial_t U^\dagger\ket{\psi'} &= UHU^\dagger\ket{\psi'}\\
	Ui\hbar \left( \dot{U}^\dagger\ket{\psi'} + U^\dagger\partial_t \ket{\psi'}\right) &= UHU^\dagger\ket{\psi'}\\
	i\hbar \partial_t \ket{\psi'} &= UHU^\dagger\ket{\psi'} - i\hbar U\dot{U}^\dagger\ket{\psi'} \\
	H' &= UHU^\dagger -i\hbar U\dot{U}^\dagger
\end{align*}
For the interaction rep $U = e^{i\frac{H_0}{\hbar} t} = e^{-i\frac{\omega_0}{2}\sigma_z t}$ and for the field interaction rep $U = e^{i\frac{\omega}{2}\sigma_z t}$
\subsection{Semiclassical dressed states}
Working in the field interaction representation. If we assume $\Omega_0$ is time independant:
\begin{align*}
	\tilde{H} &= -\frac{\hbar \delta}{2}\sigma_z + \frac{\hbar}{2}\Re \Omega_0 \sigma_x +\frac{\hbar}{2}\Im \Omega_0 \sigma_y \\
	\tilde{H} \ket{\pm} &= E_\pm \ket{pm}
\end{align*}
Solving for the eigenvalues:
\begin{align*}
	E_\pm &= \pm \frac{\hbar}{2}\sqrt{\Omega_0^2 + \delta^2} \\
	E_\pm &= \pm \frac{\hbar}{2}\Omega \\
	\ket{+} &= \sin\theta\ket{1} + \cos\theta \ket{2} \\
	\ket{-} &= \cos\theta\ket{1} - \sin\theta\ket{2}
\end{align*}
If we assume $\Omega_0$ is real:
\begin{align*}
	\tan\theta &= \frac{\Omega_0}{\Omega + \delta}
\end{align*}
(Alternatively $\tan2\theta = \frac{\Omega_0}{\delta}$). \\
If we look at the case where $\delta =0$, we can see $\tan\theta = 1$, so $\theta = \frac{\pi}{4}$, so:
\begin{align*}
	\ket{+} &= \frac{1}{\sqrt{2}} (\ket{1} + \ket{2}) \\
	\ket{-} &= \frac{1}{\sqrt{2}} (\ket{1} - \ket{2})
\end{align*}
So our free evolution becomes:
\begin{align*}
	\ket{\psi} c_+(0) e^{-i\frac{\Omega_0}{2} t}\ket{+} + c_-(0) e^{i\frac{\Omega_0}{2} t} \ket{-}
\end{align*}
With the initial condition $c_1(0) = 1$ and $c_2(0) = 0$ we know $c_+ = c_- = \frac{1}{\sqrt{2}}$, so:
\begin{align*}
	\ket{\psi} &= \frac{1}{\sqrt{2}} e^{-i\frac{\Omega_0}{2} t}\ket{+} + \frac{1}{\sqrt{2}} e^{i\frac{\Omega_0}{2} t} \ket{-} \\
	\ket{\psi} &= \frac{1}{\sqrt{2}} \left(e^{-i\frac{\Omega_0}{2} t}(\ket{\tilde{1}} + \ket{\tilde{2}}) +  e^{i\frac{\Omega_0}{2} t} (\ket{\tilde{1}} + \ket{\tilde{2}})\right) \\
	\ket{\psi} &= \frac{1}{\sqrt{2}} \left(e^{-i\frac{\Omega_0}{2} t}(\ket{\tilde{1}} + \ket{\tilde{2}}) +  e^{i\frac{\Omega_0}{2} t} (\ket{\tilde{1}} + \ket{\tilde{2}})\right) \\
	\ket{\psi} &= \frac{1}{\sqrt{2}}\left(\cos\frac{\Omega_0 t}{2} \ket{\tilde{1}} - i\sin\frac{\Omega_0 t}{2} \ket{\tilde{2}}\right)
\end{align*}
Which gives us exactly the same behavior of Rabi oscilations, that we see in other representations.\\
If we now pick a large detuning $|\delta| \gg \Omega_0$. We first consider positive detuning (red detuning) $\omega_0 > \omega$. We again look at the states $\ket{\pm}$:
\begin{align*}
	\Omega &= \sqrt{\delta^2 + \Omega_0^2} \\
	\Omega &= \delta\sqrt{1 + \frac{\Omega_0^2}{\delta^2}} \\
	\Omega &= \delta + \frac{\Omega_0}{2\delta}
\end{align*}
We then know $\theta\ll 1$, so $\ket{+} \approx \ket{\tilde{2}}$, and $\ket{-} \approx \ket{\tilde{1}}$. We call these bare states. Therefore in the far detuned range we have essentially just shifted the energy states:
\begin{align*}
	E_\pm &= \pm \frac{\hbar}{2}\left(\delta + \frac{\Omega_0^2}{2\delta}\right) 
\end{align*}
This corresponds to shifts by $\frac{\Omega_0^2}{4\delta}$ up/down for each state. This is called the optical stark shift. \\
If we instead have a negative detuning (blue detuning) $\omega > \omega_0$. First we can tell that the states have their orders swapped ($\ket{\tilde{1}}$ is higher energy than $\ket{\tilde{2}}$). 
We have our $\Omega \approx -\delta -\frac{\Omega_0^2}{2\delta}$. Additionally we see our angle will be $\theta \to \frac{\pi}{2}$. So then:
\begin{align*}
	\ket{+} &\approx \ket{\tilde{1}} \\
	\ket{-} &\approx -\ket{\tilde{2}} \\
\end{align*}
So we have bare states, but swapped from the red detuning. Our energies are shifted as before by $\frac{\Omega_0^2}{4|\delta|}$ with $\ket{+}$ being upshifted and $\ket{-}$ being downshifted. In terms of the initial states $\ket{1}$ and $\ket{2}$, we have our energies pulled closer together. In other words the blue-detuned case leads to a red-shift for the transition, and the red-detuned case leads to a blue-shift for the transition. \\
When $\delta$ is small we see strong mixing between our states, and when $\delta$ is large we see small amounts of mixing. \\
\subsection{Adiabatic Following}
We now consider the case of time dependant $\Omega_0$ and $\delta$. If we say our time dependance is sufficiently slow, then we have adiabatic following. \\
For a time independant following we can solve for the eigenstates of a system. A particle in an eigenstate stays in the eigenstate. For a time dependant hamiltonian, we have a set of (instantaneous) eigenstates at every time $t$, but we don't have a guarantee that we will stay in those eigenstates. \\
In adiabatic following we evolve our Hamiltonian slowly, and start in an eigenstate at the start, as a result of this we remain in the instantaneous eigenstate of the system as that eigenstate evolves. Using this in our system you can drive the system from a lower state into a higher state (Landen Zener crossing). This can sometimes be more practical than a $\pi$ pulse.
