\section{Pulsed Laser Physics}
Up until now we have focused on light sources where the light is monochromatic, and therefore we are dealing with continuous wave operation lasers.
If we relax this we will be looking at ``broad band'' (pulsed light). Here the pulse implies that the amplitude is time dependant.
An example might be a flash for flash photography. We want to focus on coherent pulses.
\subsection{Coherence}
The classic experiment demonstrating spatial coherence is Young's double slit experiment.\\
\includegraphics*[width=12cm]{5-12-1}\\
The intensity at the measurement location is given by:
\begin{align*}
	|E|^2 &= |E_1 e^{i\phi_1} + E_2 e^{i\phi_2}|^2 \\
	&= |E_1|^2 + |E_2|^2 + 2\Re{E_1 E_2^* \cos(\phi_1-\phi_2)}
\end{align*}
Which for random phases just gives us $|E|^2 = |E_1|^2 + |E_2|^2$.

Alternatiely if we create a Michaelson interferometer, we see:
\begin{align*}
	I &= \frac{1}{4}\left(|E(t)|^2 + |E(t+\tau)|^2 + 2\Re{E(t)E^*(t+\tau)}\right)
\end{align*}
We define the visability:
\begin{align*}
	V &= \frac{\expval{E_1E_2^*}}{\sqrt{\expval{|E_1|^2}\expval{|E_2|^2}}} \\
	&= |g_{ij}^{(1)}|
\end{align*}
Which is a measurement of our coherence.

For an ideal monochromatic laser we say the electric field is:
\begin{align*}
	E(t) &= E_0 \cos\omega_0 t
\end{align*}
And this gives us an intensity for our Michaelson interferometer:
\begin{align*}
	T(t,\tau) &= \frac{|E_0|^2}{4}\left(2\frac{1}{2} + 2\Re{\frac{1}{T}\int dt \cos\omega_0 t\cos\omega_0(t+\tau)}\right) \\
	T(t,\tau) &= \frac{|E_0|^2}{4}\left(1 + \cos(\omega_0\tau)\right)
\end{align*}
Where this has been averaged over the detector response time.

If we now pass our laser through a shutter of length $T$ (not the detector time):
\begin{align*}
	E_T(t)&= E_0 \cos \omega_0 t TH_T(t)
\end{align*}
We look at the spectrum before and after our shutter:
\begin{align*}
	\tilde{I}(\omega) &= |\tilde{E}(\omega)|^2
\end{align*}
Which before the shutter is:
\begin{align*}
	\tilde{I}(\omega) &= \frac{|E_0|^2}{4} 2\pi\left(\delta(\omega + \omega_0) + \delta(\omega - \omega_0)\right)
\end{align*}
And after the shutter becomes:
\begin{align*}
	\tilde{E}_T(\omega) &= \frac{E_0}{2} t\left[\sinc\frac{(\omega - \omega_0)T}{2} + \sinc\frac{(\omega + \omega_0)T}{2}\right]
\end{align*}
If we assume that $\omega_0 \gg \frac{1}{T}$ then these will be seperate peaks of width $\delta\omega\approx\frac{1}{T}$ and centers at $\omega_0$ and $-\omega_0$.

We now look at a coherent optical pulse, which will be a solution to the slowly varying envelope approximation. We recall our Maxwell equations:
\begin{align*}
	\del\cdot\bm{D} &= 0 & \del\cdot\bm{B} &= 0 \\
	\del\times\bm{E} &= -\partial_t \bm{B} & \del\times\bm{B} &= \mu_0\partial_t \bm{D} \\
	\bm{D} &= \epsilon_0 \bm{E} + \bm{P}
\end{align*}
In terms of an envelope and a carrier we can write our electric field:
\begin{align*}
	E(t) &= A(t) e^{-i\omega_0 t} + \compcon
\end{align*}
Where $A$ is our envelope function, and $\omega_0$ is our carrier frequency. For our wave traveling through space we also include a spatial dependance:
\begin{align*}
	E(z,t) &= A_0(z,t) e^{i(k_0 z - \omega_0 t)} & A_0(z,t) &= \int \frac{d\omega}{2\pi} \tilde{A}(z,\omega) e^{i[(k(\omega)- k(\omega_0)) z - (\omega-\omega_0)t}
\end{align*}
We want a wave equation so we say:
\begin{align*}
	\del\times\del\times\bm{E} &= -\partial_t \del\times\bm{B} \\
	&= -\mu_0\partial_t^2\bm{D} \\
	&= -\mu_0\partial_t^2(\epsilon_0 \bm{E} + \epsilon_0\chi^{(1)} \bm{E} + \bm{P}^{NL}) \\
	&= -\frac{1}{c^2} \partial_t^2 (1+\chi^{(1)})\bm{E} + \bm{P}^{NL})
	&= -\nabla^2\bm{E} + \del(\del\cdot\bm{E})
\end{align*}
If we set the divergencce of our polarization equal to zero (making $\del\cdot\bm{D} = \epsilon_0\del\cdot\bm{E}$). So:
\begin{align*}
	\left(\nabla^2 - \frac{n^2}{c^2}\partial_t^2\right)\bm{E} &= -\mu_0\partial_t^2\bm{P}^{NL}
\end{align*}
We reduce to a one dimensional problem along $z$ and neglect the polarization saying:
\begin{align*}
	E(z,t) &= \sum_l A_l(z,t) e^{i(k_l z - \omega_l t)} \\
	A_l(z,t) &= \int_{\text{Band l}} \frac{d\omega}{2\pi} \tilde{A}(z,\omega) e^{i[(k(\omega) - k_l)z - (\omega - \omega_l)t]}
\end{align*}
To simplify the problem we move to fourier space, where:
\begin{align*}
	\tilde{E}(z,\omega) &= \tilde{A}(z,\omega)e^{i(k(\omega)z-\omega t)} \\
	\left(\partial_z^2 + \frac{n^2}{c^2} \omega^2\right)\tilde{A}(z,\omega) e^{ik(\omega)z} &= \mu_0\omega^2\tilde{P}^{NL}(z,\omega)
\end{align*}
We now make the slowly varying envelope approximation:
\begin{align*}
	\partial_z^2\tilde{A}(z,\omega) &= \partial_z e^{ik(\omega)z}\left(\partial_z \tilde{A} + ik(\omega)\tilde{A}\right) \\
	&= e^{ik(\omega) z}\left[ik(\omega)(\partial_z \tilde{A} + ik(\omega)\tilde{A}) + (\partial_z^2\tilde{A} + ik(\omega)\partial_z \tilde{A})\right] \\
	&= e^{ik(\omega) z}\left[-k(\omega)^2\tilde{A} + 2ik(\omega)\partial_z \tilde{A}\right]
\end{align*}
Where we have used the approximation to drop the second derivitive term. Therefore:
\begin{align*}
	e^{ik(\omega)z}\left(2ik(\omega)\partial_z -\left(k^2(\omega) - \frac{n^2}{c^2} \omega^2\right)\right)\tilde{A}(z,\omega) &= \mu_0\omega^2\tilde{P}^{NL}(z,\omega)
\end{align*}
But $k(\omega) = \frac{n(\omega)\omega}{c}$ so:
\begin{align*}
	\partial_z \tilde{A}(z,\omega) e^{ik(\omega)z} &= \frac{\mu_0\omega^2}{2ik(\omega)}\tilde{P}^{NL}(z,\omega)
\end{align*}
We use this to construct $A_l$:
\begin{align*}
	\partial_t A_l &= \int_{Bl} \frac{d\omega}{2\pi} e^{\ldots} \left[\partial_z \tilde{A} + i(k(\omega) - k_l)\tilde{A}\right] \\
	&= \int_{Bl} \frac{d\omega}{2\pi} e^{\ldots} \left[\partial_z \tilde{A} + ik'_l(\omega-\omega_l)\tilde{A} + \frac{ik''_l}{2}(\omega-\omega_l)^2\tilde{A}\right]
\end{align*}
And:
\begin{align*}
	\int_{Bl} \frac{d\omega}{2\pi} e^{\ldots} \partial_z \tilde{A} &= \int_{Bl} \frac{d\omega}{2\pi}e^{\ldots} \frac{\mu_0 \omega^2}{2ik(\omega)} e^{-ik(\omega) z} \tilde{P}^{NL} \\
	&= e^{-ik_l z}\int_{Bl} \frac{d\omega'}{2\pi} e^{-i\omega' t} \frac{\mu_0 (\omega' + \omega_l)^2}{2ik(\omega' + \omega_l)} \tilde{P}(^{NL}(z,\omega' + \omega_l) \\
	&= e^{-i(k_lz- \omega_l t)} \frac{\mu_0}{2ik_l} \partial_t^2 P_l
\end{align*}
Also:
\begin{align*}
	\int\frac{d\omega}{2\pi} e^{\ldots} ik'_l(\omega- \omega_l)\tilde{A} &= - k_l'\partial_t \int \frac{d\omega}{2\pi} e^{\ldots} \tilde{A} \\
	&= -k_l'\partial_t A_l
\end{align*}
Finally:
\begin{align*}
	\int\frac{d\omega}{2\pi} e^{\ldots} i\frac{k''_l}{2}(\omega- \omega_l)^2\tilde{A} &= - \frac{ik_l''}{2}\partial_t^2 A_l
	&= -k_l'\partial_t A_l
\end{align*}
We know:
\begin{align*}
	k'_l &= \frac{dk}{d\omega}\Big|_{\omega=\omega_l} = \frac{1}{v_{gl}}
\end{align*}
Which is the group velocity, we also have a group velocity dispersion (GVD). Putting everything togetther:
\begin{align*}
	\partial_z A_l + \frac{1}{v_{gl}} \partial_t A_l + i\frac{k''_l}{2}\partial_t^2 A_l &= -\frac{i\mu_0\omega_l^2}{2k_l} P_l^{NL} e^{-ik_l z}
\end{align*}
