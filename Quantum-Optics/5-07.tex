\subsection*{Asside: Optical Isolator}
These work via the Faraday/Kerr effect. An electric field propogative from a material exposed to a DC magnetic field will experience birefringence as it passes through the medium.
Because the magnetic field is a pseudo-vector field, light propogating backwards through this material will experience the same rotation, instead of the opposite rotation. Using this along with polarizers, you can build an isolator.
\subsection{Four wave mixing}
So far we have considered $P^{NL} = \epsilon_0\chi^{(2)} E^2 + \epsilon_0\chi^{(3)} E^3$. Since $\chi^{(2)}$ appears only for non-centro-symmetric materials, so for centro-symmetric materials we will only have the $\chi^{(3)}$ term.
Interestingly interfaces will be non-centro-symmetric so there will be a $\chi^{(2)}$ nonlinearity at boundries, this can be used for something called non-linear microscopy.

Since $\chi^{(3)}$ appears for all materials it is a more fundamental non-linearity, we recall:
\begin{align*}
	\partial_z A_l + \frac{1}{v_{gl}} \partial_t A_l &= \frac{i\omega_l}{2\epsilon_0 cn_l} e^{-ik_l z} P_l^{NL}
\end{align*}
So for monochromatic fields, looking at the third order non-linearity:
\begin{align*}
	\partial_z A_l(z) &= \frac{i\omega_l \chi^{(3)}}{2cn_l} E^3 & E &= \sum_l A_l(z) e^{i(k_l z - \omega_l t)}
\end{align*}
If we pump at $\omega_1$ and $\omega_2$ (and a weak seed at $\omega_4$), then our output will be at four frequencies $\omega_1$ $\omega_2$ $\omega_3$ and $\omega_4$. Which must satisfy:
\begin{align*}
	\omega_1 + \omega_2 &= \omega_3 + \omega_4
\end{align*}
We now seek to find $P^{(3)}_3$:
\begin{align*}
	\omega_3 &= \omega_1 + \omega_2 - \omega_4 \\
	P_3^{(3)} &= 3!\epsilon_0 \chi^{(3)} A_1 A_2 A_4^* e^{i(k_1 + k_2 - k_4)z} \\
	P_4^{(3)} &= 3!\epsilon_0 \chi^{(3)} A_1 A_2 A_3^* e^{i(k_1 + k_2 - k_3)z}
\end{align*}
Turning to the case of third harmonic generation, we only have two fields, and:
\begin{align*}
	\omega_2 &= 3\omega_1 \\
	P_2^{(3)} &= \epsilon_0\chi^{(3)} A_1^3 e^{i3k_1 z}
\end{align*}
Looking at our equations for monochromatic light:
\begin{align*}
	\partial_z A_1 &= \frac{i\omega_1}{2c n_1} \chi^{(3)} e^{-ik_1 z} 3A_2 A_1^* A_1^* e^{i(k_2-2k_z)z} \\
	\partial_z A_2 &= \frac{i\omega_2}{2c n_2} \chi^{(3)} e^{-ik_2 z} A_1^3!e^{i3k_1z} 
\end{align*}
We have a $\Delta k = k_2-3k_1$ like before, so:
\begin{align*}
	\partial_z A_1 &= \frac{i3!\omega_1}{2c n_1} \chi^{(3)} e^{i\Delta k z} A_2 A_1^* A_1^* \\
	\partial_z A_2 &= \frac{i\omega_2}{2c n_2} \chi^{(3)} e^{-i\Delta k z} A_1^3
\end{align*}
We make the undepleted pump approximation here, so that:
\begin{align*}
	A_2(z) &= \frac{i\omega_2 \chi^{(3)}A_1^3}{2cn_2}e^{-i\frac{\Delta k z}{2}} \text{sinc}\left(\frac{\Delta k z}{2}\right)
\end{align*}
Which looks very similar to SHG. The key difference being that this scales with the pump cubed instead of the pump squared.

Turning back to our initial three wave mixing process, we can see, that since if $A_3$ and $A_4$ start at zero, then the derivitive must be zero, and therefore they will always remain zero.
So in (classical) four wave mixing we need a seed field to casuse it to produce light. We now make the undepleted pump approximation for $A_1$ and $A_2$, so:
\begin{align*}
	\partial_z A_3 &= g_3 A_4^* e^{i\Delta kz} \\
	\partial_z A_4 &= g_4 A_3^* e^{i\Delta kz}
\end{align*}
Which are identical to our equations for the parametric amplifier. If we have perfect phase matching ($\Delta k =0$), then:
\begin{align*}
	\partial_z A_3 &= g_3 A_4^* \\
	\partial_z A_4 &= g_4 A_3^* 
\end{align*}
Which gives us:
\begin{align*}
	\partial_z^2 A_3 &= g_3g_4^* A_3 \\
	\partial_z^2 A_4 &= g_4g_3^* A_4
\end{align*}
We can construct an OPO (optical parametric oscilator) in analogy to a laser, with a non-linear material in a cavity, being pumped by another optical field (or pair of optical fields). This essentially creates laser-like light.
These are often called ``DOPA''s or ``NOPA''s (these are non-colinear and tunable).

If we put in just a single field $\omega_1$ Then we can see:
\begin{align*}
	P_1^{(3)} &= \epsilon_0 \chi^{(3)} 6A_1A_1A_1^* \\
	\partial_z A_1 &= \frac{i3\omega_1 \chi^{(3)}}{cn_1}I_1A_1
\end{align*}
But this can be thought of as a modification to the index of refraction when we move back to the temporal domain. This is because the $i\omega$ acts like a time derivitve. Therefore we have an intensity dependant refractive index.
We call this effect self-phase modulation.
