\subsection*{Note from Homework}
Instead of our earlier definition of the Bloch vector:
\begin{align*}
	\rho_{12} &= \frac{1}{2}(u + iv) \\
	w &= \rho_{22} - \rho_{11}
\end{align*}
We can instead say:
\begin{align*}
	\bm{R} &= \Tr{\rho\bm{\sigma}}
\end{align*}
But this definition has a different sign attached to $v$ and $w$.
\subsection{Inhomogeneous broadening}
So far we have assumed that all the two level systems we are looking at have the same transitional frequency $\omega_0$. This is known as homogeneous broadening. I.e. we look at the width of our absorption graph, all atoms have the same linewidth $\gamma$.
What we usually see in real systems is that these transition frequencies vary from atom to atom. This can be caused by many different effects.

We first consider the broadening caused by the Doppler shift from the motion of the atoms, this is known as Doppler broadening. From the perspective of a moving atom, the frequency of the incident optical field is shifted: $\omega \to \omega - \bm{k}\cdot\bm{v}$.
Therefore the detuning seen by the atom will now be $\delta = \omega_0 - (\omega - \bm{k}\cdot\bm{v})$ but we can rewrite this as $\delta = (\omega_0 + \bm{k}\cdot\bm{v}) - \omega$.
This means that from the perspective of the lab, the frequency of the transition is shifted by $\omega_0 \to \omega_0 + \bm{k}\cdot\bm{v}$. As a result of the motions of the atoms causing these shifts, we now have a distribution of frequencies $\omega_0$,
and we see then that this causes our absorption spectrum to become broader.

Looking instead at atomic ions in a crystal (e.x. color centers in diamonds). Due to imperfections in the crystal the environments around each atom is different (could be electric fields or mechanical strains).
These different local imperfections lead to different transition frequencies $\omega_0$. This distribution is typically Gaussian:
\begin{align*}
	g(\omega_0) &= \frac{1}{\sigma_\omega \sqrt{\pi}} e ^{-\frac{(\omega_0 - \bar{\omega}_0)^2}{\sigma_\omega^2}}
\end{align*}
(Note: this Gaussian is essentially single tailed, i.e. we assume $\bar{\omega}_0\gg\sigma_\omega$ so $\int_0^\infty g(\omega_0) d\omega_0 = 1$ instead of an integral from $-\infty$ to $\infty$) \\
Our absorption then follows:
\begin{align*}
	P &= \frac{N}{V}\left[\mu\rho_{21} + \text{c.c.}\right] \\
	P &= \frac{N}{V}\left[\mu\int_0^\infty\rho_{21}(\omega_0)d\omega_0 + \text{c.c.}\right] \\
	\chi_\text{inh}(\omega) &= \int_0^\infty d\omega_0 g(\omega_0) \chi(\omega,\omega_0)
\end{align*}
In other words we see that our $\chi$ now is the convolution of our distribution of $\omega_0$ over a kernel $\chi$.

For linear absorption we can see:
\begin{align*}
	\chi_\text{inh}(\omega) &= \frac{N}{V} \int_0^\infty d\omega_0 g(\omega_0) \frac{\mu^2}{\epsilon_0\hbar} \frac{\omega_0 - \omega + i\gamma}{(\omega_0 - \omega)^2 + \gamma^2} \\
	\alpha_\text{inh}(\omega) &=  \frac{\omega N \mu^2}{c V \epsilon_0\hbar} \int_0^\infty d\omega_0 g(\omega_0) \frac{\gamma}{(\omega_0-\omega)^2 + \gamma}
\end{align*}
If we say that the Gaussian is very sharp (essentially a delta function) $\sigma_\omega \ll \gamma$ then we say that our system has a Lorentzian absorption, and we call this homogeneously broadened.
If instead the Lorentzian is much sharper $\sigma_\omega \gg \gamma$ then we say our system has Gaussian absorption and is inhomogeneously broadened. Most of the time we will be somewhere in the middle of these two.
\subsection{Summary}
We used the Maxwell-Bloch equations to describe the macroscopic behavior of optical interactions.\\
We saw that the polarization played a critical role in how our system reacted to the optical field. \\
We saw that nonlinear absorption lead to absorption saturation and power broadening. \\
Finally we looked a the rate equation approximation, which required $\rho_{21}$ following $\rho_{11}$ and $\rho_{22}$ instantaneously.
\section{Laser theory}
We know that laser refers to light amplification by stimulated emission of radiation. A more accurate but unfortunate acronym for this is light oscillation through stimulated emission of radiation.
We know that our amplification or gain implies a negative absorption coefficient. Since $\alpha \propto (\rho_{11} - \rho_{22})$ this requires that $\rho_{22} > \rho_{11}$, which is known as population inversion.

We additionally need to have stimulated emission, which is governed by:
\begin{align*}
	\omega &= \frac{\gamma}{2(\delta^2 +\gamma^2)} |\Omega_0|^2 \\
	\omega &\propto I \\
	\dot{\rho}_{22} &= -\gamma_2 \rho_{22} + w(\rho_{11} - \rho_{22})
\end{align*}
The stimulated term is the final term in the last equation $-w\rho_{22}$.

We see that our atom makes a transition $\ket{2}\to\ket{1}$ which emits some light. This emitted light has the same frequency phase and propagation direction as the incident light.
This corresponds to an amplification of the incident light.

In order to build a laser we need three things:\\
1) An active medium: a system that provides gain via an atomic transition (could be liquid, solid, gas plasma, etc.) \\
2) Pumping mechanism: something to cause population inversion (could be optical, mechanical, chemical, electrical, etc.) \\
3) An optical cavity/resonator: something to sustain the oscillation and define the spatial mode (typically a pair of mirrors) \\

This will require at least a three level scheme in order to have a steady state population inversion.
In a three level scheme, we have a transition from $\ket{3}$ to $\ket{2}$ that releases energy, and we force a transition from $\ket{1}$ to $\ket{3}$ via some other process.
For four level systems we instead have a transition from $\ket{2}$ to $\ket{1}$ that releases energy, a natural transition from $\ket{1}$ to $\ket{0}$ and $\ket{3}$ to $\ket{2}$, and we pump from $\ket{0}$ to $\ket{3}$.
Four level schemes are considered more effective than three level schemes (though the original laser was a 3 level scheme).

The cavity will give us some of our spectral properties, though the linewidth of the laser is typically much sharper than the linewidth of the cavity.
\subsection{Semiclassical treatment of lasers}
We say that in our picture from last chapter $E\to P\to E$ that final transition corresponds to our stimulated emission.\\
Defining:
\begin{align*}
	E &= \frac{1}{2}\sum_n E_n(t) e^{i(\omega_n t - \phi_n)} U_n(z) + \text{c.c.}
\end{align*}
We assume that we have a traveling wave so $U_n(z) = e^{ik_n z}$. Note $\omega_n \neq k_n c$ ($k_n c$ is called the bare cavity resonance) \\
We then can say our polarization is:
\begin{align*}
	P &= \frac{1}{2}\sum_n P_n e^{-i(\omega_n t - \phi_n)} U_n(z) \\
	P_n &= \epsilon_0 \chi_n E_n
\end{align*}
In the Maxwell wave equation:
\begin{align*}
	\left(\partial_z + \frac{1}{c}\partial_t\right)E_0 &= -\frac{k}{2\epsilon_0} \Im P_0 \\
	E_0\left(\partial_z + \frac{1}{c}\partial_t\right)\phi &= -\frac{k}{2\epsilon_0}\Re P_0
\end{align*}
For a cavity:
\begin{align*}
	\partial_z E_n &= 0 &
	\partial_z \phi_n &= 0
\end{align*}
And we also have a cavity loss factor $\kappa$, so:
\begin{align*}
	\partial_t E_0 &= -\frac{1}{2}\kappa E_0 &
	\partial_t I &= -\kappa I &
	Q &= \frac{\omega}{\kappa}
\end{align*}
Finally we know:
\begin{align*}
	\omega_n\neq k_n c
\end{align*}
So our MWE becomes:
\begin{align*}
	\dot{E}_n &= -\frac{\kappa_n}{2}E_n - \frac{\omega_n}{2\epsilon_0}\Im P_n \\
	\dot{\phi}_n +\omega_n - k_n c &= -\frac{\omega_n}{2\epsilon_0}\Re \frac{P_n}{E_n}
\end{align*}
Where we get an extra term for $\phi$ because we assumed $\omega = kc$ in our original derivation of the MWE, which is no longer valid here. We can now see:
\begin{align*}
	\dot{E}_n &= -\frac{\kappa_n}{2}E_n - \frac{\omega_n}{2}\chi_n'' E_n \\
	\dot{\phi}_n + \omega_n - k_n c &= -\frac{\omega_n}{2}\chi'
\end{align*}
Therefore just looking at our intensity:
\begin{align*}
	I_n &= E_n^2 \\
	\dot{I}_n &= -\kappa_n I_n -\omega_n\chi_n''I_n \\
	\dot{I}_n &= -\kappa_n I_n +G_n I_n \\
\end{align*}
Where we say $G_n$ is our gain rate $G_n = -\omega_n\chi_n''$. Although one might assume we have large gains in steady state, we can see that instead $G_n = \kappa_n$ during any steady state operation!

We now calculate $G_n$. We work in the rate equation approximation $\gamma\gg\gamma_2$
\begin{align*}
	\tilde{\rho}_{21} &= \frac{i}{2}\Omega_0 \frac{\rho_{22} - \rho_{11}}{i\delta + \gamma}
\end{align*}
If we ignore spontaneous emission (assuming stimulated emission is much stronger than spontaneous emission):
\begin{align*}
	\dot{\rho}_{11} &= \lambda_1 - \gamma_1\rho_{11} + w(\rho_{22} - \rho_{11}) \\
	\dot{\rho}_{22} &= \lambda_2 - \gamma_2\rho_{22} - w(\rho_{22} - \rho_{11}) \\
\end{align*}
In steady state these are both zero so:
\begin{align*}
	0 &= \frac{\lambda_1}{\gamma_1} - \rho_{11} + \frac{w}{\gamma_1}(\rho_{22} - \rho_{11}) \\
	0 &= \frac{\lambda_2}{\gamma_2} - \rho_{22} - \frac{w}{\gamma_2}(\rho_{22} - \rho_{11}) \\
	\rho_{22} - \rho_{11} &= \frac{\frac{\lambda_2}{\gamma_2} - \frac{\lambda_1}{\gamma_1}}{1 + \frac{\omega}{\omega_s}} \\
	\omega_s &= \left(\frac{1}{\gamma_1} + \frac{1}{\gamma_2}\right)^{-1}
\end{align*}
This implies that for a stronger field we have less population inversion, i.e. our stimulated emission depletes our population inversion.

We now look at:
\begin{align*}
	P &= \frac{N}{V}(\mu\rho_{21} + \text{c.c.}) \\
	P &= \frac{N}{V}(\mu\tilde{\rho}_{21}e^{-i\omega_n t} + \text{c.c.}) \\
	P &= \frac{N}{V}(\mu \frac{i}{2}\Omega_0\frac{\frac{\lambda_2}{\gamma_2} - \frac{\lambda_1}{\gamma_1}}{1 + \frac{\omega}{\omega_s}} e^{-i\omega_n t} + \text{c.c.}) \\
	\chi''_n &= -\frac{N\mu^2}{V\epsilon_0\hbar} \frac{\gamma}{\delta^2+\gamma^2} \frac{\frac{\lambda_2}{\gamma_2} - \frac{\lambda_1}{\gamma_1}}{1 + \frac{w}{w_s}}
\end{align*}
