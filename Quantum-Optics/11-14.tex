\section{Three Level Systems}
\subsection{Introduction}
We say our system has a transition up from 1 to 2, and down from 2 to 3. Since the parity of the system changes due to dipole transitions, we cannot have a dipole transition from 1 to 3. 
For this we will ignore all quadrapole and above transitions. We label our transitions between 2 and 3 with a '. We also know that 3 is in a higher energy state than 1.
We say that there are two different E fields coupling the transition, since these energies can be very far detuned, or that there may be different polarizations required for each interaction.

To make the system more symmetric, we choose our energy for 2 to be 0, which means our Hamiltonian is:
\begin{align*}
	H &= H_0 + V \\
	H_0 &= \hbar \begin{pmatrix}
		-\omega_0 & & \\
			  & 0& \\
			  && -\omega_0'
		     \end{pmatrix}\\
	V &= \frac{\hbar}{2} \begin{pmatrix}
		0 & \Omega_0^* e^{i\omega t} & 0 \\
		\Omega_0 e^{-i\omega t} & 0 & \Omega_0' e^{i\omega' t} \\
		0 & \Omega_0'\ ^* e^{i\omega' t} & 0
			     \end{pmatrix}
\end{align*}
In the interaction representation:
\begin{align*}
	\ket{\psi} &= \bar{c}_1 e^{i\omega_0 t} \ket{1} + \bar{c}_2\ket{2} + \bar{c}_3 e^{i\omega_0' t}\ket{3}
\end{align*}
Whereas in the field interaction representation:
\begin{align*}
	\ket{\psi} &= \tilde{c}_1 e^{i\omega t} \ket{1} + \tilde{c}_2\ket{2} + \tilde{c}_3 e^{i\omega' t}\ket{3} \\
	\tilde{H} &= \hbar \begin{pmatrix}
		-\delta & \frac{1}{2}\Omega_0^* & 0 \\
		\frac{1}{2}\Omega_0 & 0 & \frac{1}{2}\Omega_0' \\
		0	&\frac{1}{2}\Omega_0'\ ^*   & -\delta'
			   \end{pmatrix}
\end{align*}
We call this system the $\Lambda$-type system.
\subsection{Dark states and Coherent population trapping}
If we start in a superposition of the two lower states:
\begin{align*}
	\ket{\psi(0)} &= c_1\ket{1} + c_3\ket{3}
\end{align*}
And then excite both transitions via optical fields, we can say:
\begin{align*}
	|c_2|^2 &= |c_{1\to2} + c_{3\to2}|^2
\end{align*}
If we excite this in such a way that $c_{1\to2} = - c_{3\to2}$ then clearly:\
\begin{align*}
	|c_2|^2 &= 0
\end{align*}
So we have no transition in this case. We call the state that causes this transition to be zero a ``dark state''. These state typically show up in experments (because it is an eigenstate state under the time evolution of our system).
In this dark state we find the atom trapped in the lower state. We refer to this as coherent population trapping.
We label this state $\ket{D}$. If $\delta = \delta'$ then this state is:
\begin{align*}
	\ket{D} &= \frac{1}{\sqrt{|\Omega_0|^2 + |\Omega_0'|^2}} (\Omega_0'\ket{1} - \Omega_0\ket{3})
\end{align*}
It is easy to show that $\ket{2}\tilde{V}\ket{D} = 0$, which clearly makes our Hamiltonian diagonal.
This is called a Raman resonance or two photon resonance. This transition corresponds to a virtual energy level shifted downn by $\delta$. If we look at this with our field interaction hamiltonian explicitly:
\begin{align*}
	\tilde{H}\ket{D} &= \hbar\begin{pmatrix}
		-\delta\Omega_0' \\
		0 \\
		\delta'\Omega_0
			    \end{pmatrix}
\end{align*}
And if $\delta =\delta'$ then clearly $\tilde{H}\ket{D} = -\hbar \delta D$. This is then an eigenstate, and as a result a system that starts in this state must always remain in this state.

We now take two ``Trivial'' limits.\\
1) $\ket{\psi_A} = \ket{1}$, which means $\Omega_0=0$. Clearly since our optical field is only being applied to the side of the system with no occupation, then the $\ket{2}$ state must remain dark. \\
2) $\ket{\psi_B} = \ket{3}$, which means $\Omega_0'=0$ and by the same logic this must be dark.

With these two in hand, we now vary our fields $\Omega_0$ and $\Omega_0'$ slowly enough to maintain adiabatic follow $\ket{D}$. Doing this you can change from $\ket{1}$ to $\ket{3}$ without ever entering $\ket{2}$.
We call this Stimulated Raman adiabatic passage (STIRAP). In this process you will gradually turn $\Omega_0'$ on and off, and then $\Omega_0$ on and off, with a period of time where these two overlap.
This is in opposition to the intuitive pulse sequence which is the opposite order, but makes use of the $\ket{2}$ state.

Since we already have an eigenstate $\ket{D}$, we want to look at a state orthogonal to $\ket{D}$. We propose a ``Bright'' state:
\begin{align*}
	\ket{B} &= \frac{1}{\sqrt{|\Omega_0|^2 + |\Omega_0'|^2}} (\Omega_0^*\ket{1} + \Omega_0'\ ^*\ket{3})
\end{align*}
This bright state in face is strongly coupled to the $\ket{2}$ state, so it is not an eigenstate of our system. It has an effective Rabi frequency $\Omega = \sqrt{\Omega_0|^2 + \Omega_0'|^2}$.
This is effectively a two level system, with $\ket{B}$ coupled to $\ket{2}$ and a tertiary state $\ket{D}$ which is not coupled to our system at all! We have a new picture now:
\begin{align*}
	\tilde{H}_D &= \hbar \begin{pmatrix}
		-\delta & 0 & 0 \\
		0 & 0 & \frac{\Omega}{2} \\
		0 & \frac{\Omega}{2} & -\delta
			     \end{pmatrix}
\end{align*}
Where our basis states are (in order) $\ket{D},\ket{2},\ket{B}$. We then have eigenstate $\ket{+}$, $\ket{-}$ and $\ket{D}$.

If we have spontaneous emission from 2, then sometimes we will decohere back to the $\ket{B}$ state when excited, and therefore over time we should find ourselves overwhelmingly transitioning to the $\ket{D}$ state.
We refer to this phenomena as optical pumping.
\subsection{Density Matrix for 3-level systems}
We have:
\begin{align*}
	i\hbar \partial_t \rho  &= [H,\rho]
\end{align*}
We identify components of our density matrix, $\rho_{12},\rho_{32}$ represent our dipole coherence, $\rho_{11},\rho_{22},\rho_{33}$ are the populations, and $\rho_{13}$ is the non-radiative coherence/spin coherence. $\rho_{13}$ is related to the $\ket{D}$ state.
