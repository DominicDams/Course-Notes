We can tell that these systems will arrise from things with Hamiltonians of the form $\hat{H} = \hat{a}_1\hat{a}_2 + \hercon$

We can say:
\begin{align*}
	\hat{S}^\dagger \hat{q}_\pm \hat{S} &= e^{\mp r} \hat{q}_\pm \\
	\hat{S}^\dagger \hat{p}_\pm \hat{S} &= e^{\pm r} \hat{p}_\pm \\
\end{align*}

Looking now at the Wigner representation:
\begin{align*}
	W_{12}(q_1,p_1;q_2,p_2) &= \frac{1}{4\pi^2} \int dq'\int dq'' \bra{q_1 - \frac{q'}{2}}\bra{q_2 - \frac{q''}{2}} \ket{\xi}_{12}\bra{\xi}_{12}\ket{q_1 + \frac{q'}{2}}\ket{q_2 + \frac{q''}{2}} e^{ip_1q'} e^{ip_2q''}
\end{align*}
In order to evaluate this we want to look at the $q_\pm$ basis:
\begin{align*}
	\bra{q_+}\bra{q_-}\ket{\xi} &= \psi_0(q_+ e^{-r})\psi_0(q_- e^r)
\end{align*}
Looking at the quadratures in terms of creation and annihilation operators:
\begin{align*}
	\hat{q}_\pm &= \frac{\hat{a}_1 \pm \hat{a}_2}{2} + \frac{\hat{a}_1^\dagger \pm \hat{a}_2^\dagger}{2}
\end{align*}
We can recognize that this is the same as a 50:50 beamsplitter transformation. We can also observe that the squeezed state seems to be simply the two seperate squeezed states (with perpendicular squeezing) combined at a beamsplitter.
Interestingly, the two seperate states that we have combined are not entangled, but the combined state is entangled. From this we can see that entanglement is a product of the mode basis that we are looking at the system with.

We can use our beamsplitter here to determine the Wigner function by looking at this as a change of basis for our Wigner function:
\begin{align*}
	q_1 &= \frac{q_+ + q_-}{\sqrt{2}} \\
	q_2 &= \frac{q_+ - q_-}{\sqrt{2}} \\
	p_1 &= \frac{p_+ + p_-}{\sqrt{2}} \\
	p_2 &= \frac{p_+ - p_-}{\sqrt{2}}
\end{align*}
So then:
\begin{align*}
	\bra{q_1}\bra{q_2}\ket{\xi} &= \psi_0(\frac{q_1 + q_2}{\sqrt{2}} e^{-r})\psi_0(\frac{q_1 - q_2}{\sqrt{2}} e^r)
\end{align*}

\subsection{Beamsplitter transoformations}
We now look at how our Wigner function transforms with a pair of general two mode states in a beamsplitter. We first note that a beamsplitter transformation is equivalent to a mode basis change.
After a 50:50 beamsplitter we transform our creation and annihilation operators:
\begin{align*}
	\hat{a}_1 &l\to \frac{1}{\sqrt{2}} (\hat{a}_3 -\hat{a}_4) \\
	\hat{a}_1 &l\to \frac{1}{\sqrt{2}} (\hat{a}_3 -\hat{a}_4) \\
	\hat{a}_3 &= \frac{1}{\sqrt{2}} (\hat{a}_1 +\hat{a}_2) \\
	\hat{a}_4 &= -\frac{1}{\sqrt{2}} (\hat{a}_1 -\hat{a}_2) \\
\end{align*}
And similarly the quadratures transfom as:
\begin{align*}
	\hat{q}_3 &= \frac{\hat{q}_1 + \hat{q}_2}{\sqrt{2}}
	\hat{q}_4 &= \frac{\hat{q}_1 - \hat{q}_2}{\sqrt{2}}
\end{align*}
And our Wigner functions become:
\begin{align*}
	W_{12}(q_1,p_1;q_2,p_2) &\to W_{34}(q_3,p_3;q_4,p_4) \\
	W_{12}(q_1,p_1;q_2,p_2) &\to W_{12}(\frac{\hat{q}_3 - \hat{q}_4}{\sqrt{2}},\frac{\hat{p}_3 - \hat{p}_4}{\sqrt{2}},\frac{\hat{q}_3 + \hat{q}_4}{\sqrt{2}},\frac{\hat{p}_3 + \hat{p}_4}{\sqrt{2}})
\end{align*}
A Wigner function of two arbitrary modes being seperable:
\begin{align*}
	W_{nm}(q_m,p_m;q_n,p_n) &=W_m(q_m,p_m)W_n(q_n,p_n)
\end{align*}
Implies that the state is seperable!

If we combine a state $\ket{\psi}$ with a coherent state $\ket{\alpha}$ on a (genera) beamsplitter we find:
\begin{align*}
	W_\text{in}(q_1,p_1;q_2,p_2) &= W_\psi(q_1,p_1)W_\alpha(q_2,p_2) \\
	W_\text{out}(q_3,p_3;q_4,p_4) &= W_\psi(tq_3 - rq_4,tp_3-rp_4)W_\alpha(rq_3+tq_4,rp_3 +tp_4)
\end{align*}
And we can see:
\begin{align*}
	W_3(q_3,p_3) &= \int dq_4\int dp_4 W_\text{out}(q_3,p_3;q_4,p_4) \\
	W_3(q_3,p_3) &= \int dq_4\int dp_4 W_\psi(tq_3 - rq_4,tp_3-rp_4)W_\alpha(rq_3+tq_4,rp_3 +tp_4)
\end{align*}
If we choose $t\approx 1$, $r\ll 1$, then $W_\alpha(rq_3+tq_4,rp_3 +tp_4) \approx \delta( rq_3 + t q_4 -q_0)\delta(rp_3 + tp_4 - p_0)$, then we see:
\begin{align*}
	W_3(q_3,p_3) &\approx W_\psi\left( q_3 - \frac{r}{t^2} (q_0-rq_3), t\left(p_3 - \frac{r}{t^2} (p_0-rp_3)\right)\right) \\
	W_3(q_3,p_3) &\approx W_\psi\left( q_3 - \frac{r}{t^2} q_0), p_3 - \frac{r}{t^2} p_0\right)
\end{align*}
Which gives as a mechanism to displace our input state in phase space.

If we now consider instead inputting a single photon state and vaccuum:
\begin{align*}
	\ket{\psi}_\text{in} &= \ket{1}\ket{0} \\
	\ket{\psi}_\text{out} &= \frac{\ket{1}\ket{0} + \ket{0}\ket{1}}{\sqrt{2}}
\end{align*}
Which is an entangled state. Similarly for any other photon number state with vaccuum we find this is still an entangled state.
We can define a class of states that has a singular P function, then mixing that with vaccuum (or some other states).
We say that Gaussian states are states with Gaussian Wigner functions (this includes the vaccuum state, coherent states, squeezed states, and thermal states).
It turns out that using Gaussian states and Gaussian operations (operations that leave Gaussian states Gaussian), you cannot perform quantum computations.
The motivation is that these are very easy to classicaly simulate, and therefore shouldn't be able to do quantum computation with collapsing the complexity class QP into P.

\subsection{Photodetection}
We have direct detection which measures:
\begin{align*}
	\int d^2 r \int dt' \expval{\hat{E}^-\hat{E}^+}
\end{align*}
There are different detectors we will use for different wavelengths, with Si detectors being useful for visible light, and InGaAs detectors being useful for NIR.
For simplicity we assume our spatial and temporal response functions will be tophat functions here.

\subsection{Balanced Homodyne}
We additionally can use a technique called balanced homodyne detection to measure light.
In this we have a beamsplitter with an unkown state and a local oscilator incident on it, and then we take the difference between the currents measured on the output of the beamsplitter. What we measure is then:
\begin{align*}
	\expval{i_3  - i_4} &= \expval{\hat{a}_3^\dagger\hat{a}_3 - \hat{a}_4^\dagger\hat{a}_4} \\
	\expval{i_3  - i_4} &= \frac{1}{2}\expval{(\hat{a}_1^\dagger +\hat{a}_2^\dagger)(\hat{a}_1 + \hat{a}_2) - (\hat{a}_1^\dagger - \hat{a}_2^\dagger)(\hat{a}_1 - \hat{a}_2)} \\
	\expval{i_3  - i_4} &= \expval{\hat{a}_1^\dagger\hat{a}_2 + \hat{a}_2^\dagger\hat{a}_1} \\
	\expval{i_3  - i_4} &= |\alpha| \bra{\psi}(\hat{a}_2 e^{-i\theta} + \hat{a}_2^\dagger e^{i\theta}\ket{\psi}
\end{align*}
