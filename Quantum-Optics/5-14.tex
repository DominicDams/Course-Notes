We now move into the ``pulse reference frame''
\begin{align*}
	z' &= z & \tau &= t- \frac{z}{v_g}
	\partial_z &- \partial_{z'} -\frac{1}{v_g}\partial_\tau & \partial_t &= \partial_\tau
\end{align*}
So we have our new set of equations:
\begin{align*}
	\partial_{z'} A_l - \frac{1}{v_g} \partial_\tau A_l + \frac{1}{v_g}\partial_\tau A_l + i\frac{k''_l}{2}\partial_\tau^2 A_l &= -\frac{i\mu_0\omega_l^2}{2k_l}P_l^{NL} e^{-ik_l z} \\
	\partial_{z'} A_l + i\frac{k''_l}{2}\partial_\tau^2 A_l &= -\frac{i\mu_0\omega_l^2}{2k_l}P_l^{NL} e^{-ik_l z} \\
\end{align*}
We now consider second harmonic generation with a pulse as the input:
\begin{align*}
	P_2^{(2)} &= \epsilon_0\chi^{(2)} A_1^2 e^{i2k_1 z} \\
	\partial_{z'} A_2 + i\frac{k''_l}{2}\partial_\tau^2 A_2 &= -\frac{i\mu_0\omega_2^2}{2k_2}P_2^{NL} e^{-ik_2 z} \\
	\partial_{z'} A_2 + i\frac{k''_l}{2}\partial_\tau^2 A_2 &= -\frac{i\omega_2^2\chi^{(2)}}{2c^2k_2}A_1^2 e^{-i(k_2-2k_1)z}
\end{align*}
If we assume phase matching we see:
\begin{align*}
	\partial_{z'} A_2 + i\frac{k''_l}{2}\partial_\tau^2 A_2 &= -\frac{i\omega_2\chi^{(2)}}{2cn_2}A_1^2
\end{align*}
Which is the same as before other than an additional term corresponding to the group velocity dispersion of the material. We call this spreading based on frequency ``chirp''.
There are a couple effects here. First the different propogation velocities of the different frequencies causes temporal spread in the generated light , called ``group velocity walkoff''.

We now look at what's known as a transform limited pulse. If we look at the spectrum of our light we have some bandwidth $\Delta\omega$ and the time domain will have a bandwidth $\tau$.
Our transform limited pulse will go as $\tau \approx \frac{1}{\Delta\omega}$, this is the smallest bandwidth we can have, and adding a phase that depends quadraticly on frequency will cause us to have a longer pulse.
We can also define a coherence length $L = nc\tau$. If the coherence length is shorter than the material we expect to need to deal with dispersive effects, and if it isn't, we don't.

We consider self phase modulation:
\begin{align*}
	P^{(3)}_1 &= 6\epsilon_0\chi^{(3)} A_1^2A_1^* e^{ik_1 l} \\
	\partial_{z'} A_1 + i\frac{k''_1}{2}\partial_\tau^2 A_1 &= -\frac{i3\mu_0\omega_1}{n_1 c}|A_1|^2 A_1 \\
	\left[\partial_{z'} + i\left(\frac{k_1''}{2}\partial_\tau^2 + \Gamma|A_1|^2\right)\right]A_1 &= 0
\end{align*}
This is example of a non-linear Schroedinger equation (with time and space swapped).
We can see that the $\Gamma |A_1|^2$ term leads to an intensity dependant refractive index.

If we consider the case where the dispersion is too weak for consideration, we then have:
\begin{align*}
	\partial_{z'}A_1 &= - i\Gamma|A_1|^2A_1 \\
	\partial_{z'} |A_1|^2 &= A_1^*\partial_{z'} + \compcon \\
	&= A_1^*(-i\Gamma|A_1|^2A_1) + \compcon \\
	&= -i\Gamma|A_1|^4 + \compcon \\
	&= 0
\end{align*}
So the only variation must be in the phase, so we can say:
\begin{align*}
	A_1 &= |A_1| e^{-i\Gamma|A_1|^2 z'}
\end{align*}
This phase modulation will shift different parts of our pulse in differnt ways, where the regions where the magnitude is linear in time have a frequency shift. We call this a time lense.
Since self-phase modulation allows us to broaden the bandwidth, we can then shorten the pulse length.

We will have soliton soltions to our non-linear Schroedinger equation when $\partial_\tau^2 A_1 = -\frac{2\Gamma}{k_1''} |A_1|^2$. If we propose as an ansatz:
\begin{align*}
	A_1 &= \text{sech}
\end{align*}
Which it turns out implies that $\Gamma$ must be negative to have soliton formation

We now look at cross phase modulation. We have two pulses going into our material (one weak and one strong). We can see that our equations of motion are:
\begin{align*}
	\partial_{z'} A_1 + i\frac{k''_1}{2}\partial_\tau^2 A_1 &= -i\Gamma_1 |A_1|^2A_1 - 2i\Gamma_1|A_2|^2A_1 \\
	\partial_{z'} A_2 + i\frac{k''_2}{2}\partial_\tau^2 A_2 &= -i\Gamma_2 |A_2|^2A_2 - 2i\Gamma_2|A_1|^2A_2
\end{align*}
If $A_2\gg A_1$ we can use $A_1$ to modulate the phase on $A_2$. Our equation of motion is (neglecting self phase modulation because cross phase can be much larger, and neglecting dispersion):
\begin{align*}
	\partial_{z'} A_2 &= -2i\gamma_2|A_1|^2 A_2 \\
	A_2 &\to A_2 e^{i\phi_2}
\end{align*}
Which only works well when we have group velocity matching.
