From our non-linear material we can observe three different ``3-wave mixing'' process, SHG, SFG, and DFG (sometimes called optical parametric amplification). These are governed by our coupled wave equations.

We could (but will not for time) concider the case of a depleted pump. In this case we must obey conservation of energy, so:
\begin{align*}
	n_3 A_3^*\partial_z A_3 + \compcon &= - n_1\left[A_1\partial_z A_1^* + \compcon\right] \\
	n_3 |A_3|^2  + n_1 |A_1|^2 &= n_1 A_1(0)
\end{align*}
So then:
\begin{align*}
	\partial_z |A_3|^2 &= \frac{-i\omega_1\chi^{(2)}}{n_1 c} \left[A_(0)^2 - |A_3|^2\right] e^{i(2\phi_1-  \phi_3)}
	|A_3| &= A_{10}\tanh -i\int \frac{\omega_1\chi^{(2)}}{b_1 c} e^{i\Delta\phi}
\end{align*}

If we repeat our undepleted calculation for SFG we find:
\begin{align*}
	|A_3|^2 &= \Big|\frac{\Omega_3\chi^{(2)}}{cn_3}\Big|^2|A_1|^2 \sinc^2 \frac{\Delta k z}{2}
\end{align*}
\subsection{DFG/OPA/phase sensative amplifier}
In DFG we start with $\omega_3$ and $\omega_1$ and generate at the difference frequency $\omega_2$ and more of the lower frequency light $\omega_1$.
This can be useful if you have a strong laser at $\omega_3$, a weak laser at $\omega_1$ and want a strong laser at $\omega_1$.

We start by making the undepleted pump assumption for $A_3$. We then have:
\begin{align*}
	\partial_z A_1 &= \frac{i\omega_1\chi^{(2)}}{n_1 c} e^{i\Delta k z} A_3 A_2^* \\
	\partial_z A_2 &= \frac{i\omega_2\chi^{(2)}}{n_2 c} e^{i\Delta k z} A_3 A_1^* \\
	g_1 &= \frac{\omega_1 \chi^{(2)}}{n_1 c} A_3 & g_2 &= \frac{\omega_2\chi^{(2)} A_3}{n_2 c} \\
	\partial_z A_1 &= ig_1 e^{i\Delta k z} A_2^* \\
	\partial_z A_2 &= ig_2 e^{i\Delta k z} A_1^*
\end{align*}
With initial conditions: $A_1(0) = A_{10}$ and $A_2(0) = 0$. Clearly if $A_{10} = 0$ we get no output (which will not be true in the quantum case).

We assume perfect phase matching $\Delta k=0$, then:
\begin{align*}
	\partial_z A_1 &= ig_1 A_2^* \\
	\partial_z A_2 &= ig_2 A_1^* \\
	\partial_z^2 A_1 &= ig_1 \partial_z A_2^* \\
	&= ig_1 (-ig_2^* A_1) \\
	&= g_1 g_2^* A_1 \\
	&= \frac{\omega_1\omega_2|\chi^{(2)}|^2 |A_3|^2}{n_1n_2c^2}A_1 \\
	&= g^2A_1 \\
	A_1(z) &= a_{1+} e^{gz} + a_{1-} e^{-gz} \\
	A_2(z) &= a_{2+} e^{gz} + a_{2-} e^{-gz}
\end{align*}
In order to match our initial conditions:
\begin{align*}
	a_{2+} + a_{2-} &= 0 \\
	a_{1+} + a_{1-} &= A_{10}
\end{align*}
So:
\begin{align*}
	A_2(z) &= 2a_{2+} \sinc(gz)
\end{align*}
And using our derivitive and the linear independance of solutions we can say:
\begin{align*}
	A_1(z) &= A_{10}\cosh(gz) \\
	A_2(z) &= \frac{ig_2A_{10}}{g}\sinc(gz) 
\end{align*}
If we concider taking $gL\gg 1$ then we see:
\begin{align*}
	A_1(L) &\approx A_{10} e^{gL} \\
	A_2(L) &\approx ig \frac{cn_1}{\omega_1\chi^{(2)}} \frac{A_{10}}{A_3} e^{gL}
\end{align*}
We now turn our attention to ``degenerate'' OPA, where $\omega_1=\omega_2$:
\begin{align*}
	A_2(L) &= i\frac{g_2 A_{10}}{g} \sinc gL \\
	&= iA_{10}\sqrt{\frac{n_1\omega_2}{n_2\omega_1}} \arg(A_3\chi^{(2)})\sinc gL
\end{align*}
This accumulated phase can cause the ouput field to experience interference! We can explicitly see this in degenerate OPA.
