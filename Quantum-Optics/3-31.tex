\chapter{Spring Term, Brian Smith}
\section*{Overview}
This quarter we will look into interactions betweeen quantum light and quantum matter. This includes cavity QED, non-linear optics, spontaneous emission, etc. Primarily we will cover non-linear optics.
If time permits we will deal with the quantum treatment of gain.
\section{Light Matter Interactions}
\subsection{Maxwell Equations}
We start by looking at the Maxwell equations:
\begin{align*}
	\del\cdot\bm{E} &= \frac{\rho_f}{\epsilon_0} & \del\cdot\bm{B} &= 0 \\
	\del\times\bm{E} &= -\partial_t \bm{B} & \del\times\bm{B} &= \mu_u\bm{J}_f + \mu_0\epsilon_0 \partial_t \bm{E}
\end{align*}
And because of linearity it sufices to look at just point charges, so:
\begin{align*}
	\rho(\bm{x},t) &= \sum_j q_j \delta(\bm{x}-\bm{x}_j(t)) \\
	J(\bm{x},t) &= \sum_j q_j \bm{v}_j(t) \delta(\bm{x}-\bm{x}_j(t)) & \bm{v}_j(t) &= \frac{d\bm{x}_j}{dt}
\end{align*}
And we finally need the force law:
\begin{align*}
	m_j\frac{d^2\bm{x}_j}{dt^2} &= q_j(\bm{E}(\bm{x}_j,t) + \bm{v}_j(t)\times\bm{B}(\bm{x}_j,t))
\end{align*}
In order to simplify this we write $\bm{E}$ and $\bm{B}$ in terms of parallel and perpendicular components:
\begin{align*}
	\bm{E} &= \bm{E}_\perp + \bm{E}_\parallel \\
	\del\cdot\bm{E}_\perp &= 0 & \del\times\bm{E}_\parallel &= 0
\end{align*}
And similarly for $\bm{B}$. Plugging this into Maxwell's equations:
\begin{align*}
	\del\cdot\bm{E}_\parallel &= \frac{\rho_f}{\epsilon_0} & \del\cdot\bm{E}_\perp &= 0 \\
	\bm{B}_\parallel &= 0 & \del\cdot\bm{B}_\perp &= 0\\
	\del\times\bm{E}_\parallel &= 0 & \del\times\bm{E}_\perp &= -\partial_t\bm{B}_\perp \\
	\del\times\bm{E}_\perp &= \mu_0 \bm{J} + \mu_0\epsilon_0\partial_t(\bm{E}_\perp + \bm{E}_\parallel)
\end{align*}
We immediately find:
\begin{align*}
	\bm{E}_\parallel(\bm{x},t) &= \sum_j \frac{q_j (\bm{x} - \bm{x}_j(t))}{4\pi\epsilon_0 |\bm{x} - \bm{x}_j(t)|^3}
\end{align*}
Which is just the Coulomb field. Therefore our longitudinal field is the ``static'' Coulomb field. In order to deal with quickly moving times, we need to deal with the retarded time in this equation.
If we look in the far field limit, then this will go to zero, so both longitudinal fields will be zero.

We now look at the transverse fields (which correspond to radiation fields), these are the components that will survive far from the charges. We intoduce our $\bm{A}$ and $\phi$ fields(working in the Coulomb gauge):
\begin{align*}
	\bm{E} &= -\del\phi -\partial_t\bm{A} \\
	\bm{B} &= \del\times\bm{A} \\
	\del\cdot\bm{A} &= 0\\
\end{align*}
And so:
\begin{align*}
	\del\cdot\bm{E} &= -\nabla^2\phi -\partial_t (\del\cdot\bm{A} \\
	-\nabla^2\phi &= \frac{\rho}{\epsilon_0} \\
	\del\cdot\bm{B} &= \del\cdot\del\times\bm{A} = 0 \\
	\del\times\bm{E} &= -\del\times\del\phi - \partial_t \del\times\bm{A} \\
	- \partial_t \del\times\bm{A} = -\partial_t \bm{B} \\
	\del\times\bm{B} &= \mu_0\bm{J} + \mu_0\epsilon_0\partial_t(-\del\phi -\partial_t\bm{A} \\
	\del\times\del\times\bm{A} &= \mu_0\bm{J} -\mu_0\epsilon_0 (\partial_t \del\phi - \partial_t^2 \bm{A}) \\
	-\nabla^2\bm{A} + \del(\del\cdot\bm{A}) + \mu_0\epsilon_0\partial_t \bm{A} &= \mu_0\bm{J} -\frac{1}{c^2}\partial_t \del\phi \\
	-\nabla^2\bm{A} + \mu_0\epsilon_0\partial_t \bm{A} &= \mu_0\bm{J} -\frac{1}{c^2}\partial_t \del\phi
\end{align*}
Where the last equation is a wave equation for $\bm{A}$ with sources $-\mu_0\bm{J}_f$ and $\frac{1}{c^2}\partial_t\del\phi$. We break up $\bm{A}$ and $\bm{J}$ into parallel and perpendicular components.
Just like with the $\bm{B}$ field the parallel field will be zero (this is a given in the Coulomb gauge). We also recognize that $\del\phi = -\bm{E}_\parallel$, therefore we can say:
\begin{align*}
	\left[\nabla^2 - \frac{1}{c^2}\partial_t^2\right] \bm{A}_\perp &= -\mu_0\bm{J}_\perp \\
	0 &= \mu_0\bm{J}_\parallel + \frac{1}{c^2} \partial_t\bm{E}_\parallel
\end{align*}
The second equation corresponds to a continuity equation for charge, while the first equation gives us a wave equation for transverse waves. Therefore our radiation in the far field only depends on the transverse motion of the charges.

\subsection{Hamiltonian for light matter interaction}
Our total Hamiltonian for the system of charges and fields will be:
\begin{align}
	H &= \sum_j \left(\frac{m_j v_j^2}{2} + \frac{\epsilon_0}{2}\int d^3\bm{x} \left[|\bm{E}(\bm{x},t)|^2 + c^2 |\bm{B}(\bm{x},t)|^2\right]\right)
\end{align}
We can break this into components so:
\begin{align*}
	H_{\text{em}\perp} &= \frac{\epsilon_0}{2}\int d^3x \left[|\bm{E}_\perp(\bm{x},t)|^2 + |\bm{B}_\perp(\bm{x},t)|^2\right]
\end{align*}
But we treated this last term, so we can then choose a set of plane wave modes labeled by $\bm{k},\sigma$ and say:
\begin{align*}
	\bm{u}_{\bm{k},\sigma}(\bm{x},t) &= \bm{e}_{\bm{k},\sigma} e^{i(\bm{k}\cdot\bm{x} - \omega_{\bm{k}} t)} \\
	\bm{E}_\perp &= \sum_{\bm{k},\sigma} \alpha_{\bm{k},\sigma} \bm{e}_{\bm{k},\sigma} e^{i(\bm{k}\cdot\bm{x} - \omega_k t)} + \compcon \\
	\bm{A}_\perp &= \sum_{\bm{k},\sigma} \alpha_{\bm{k},\sigma} \frac{\bm{e}_{\bm{k},\sigma}}{i\omega_k} e^{i(\bm{k}\cdot\bm{x} - \omega_k t)} + \compcon
\end{align*}
So then:
\begin{align*}
	\hat{H}_{\text{em}\perp} &= \sum_{\bm{k},\sigma} \hbar\omega_k \left(\hat{a}_{\bm{k},\sigma}^\dagger\hat{a}_{\bm{k},\sigma} + \frac{1}{2}\right)
\end{align*}
We now look to find the longitudinal Hamiltonian:
\begin{align*}
	H_{\text{em}\parallel} &= \frac{\epsilon_0}{2}\int d^3x \left[|\bm{E}_\parallel(\bm{x},t)|^2\right] \\
	H_{\text{em}\parallel} &= \sum_{i,j}\frac{\epsilon_0}{2}\int d^3x \frac{q_i q_j}{16\pi^2\epsilon_0^2} \frac{(\bm{x} - \bm{x}_i(t))\cdot(\bm{x} - \bm{x}_j(t))}{|\bm{x} - \bm{x}_i(t)|^3|\bm{x} - \bm{x}_j(t)|^3}
\end{align*}
But we immediately see that the self energy term diverges (if $i=j$ we have an integral $\propto \int \frac{dr}{r^2}$). In order to have a sensible energy we drop these terms, so:
\begin{align*}
	H_{\text{em}\parallel} &= \sum_{i\neq j}\frac{\epsilon_0}{2}\int d^3x \frac{q_i q_j}{16\pi^2\epsilon_0^2} \frac{(\bm{x} - \bm{x}_i(t))\cdot(\bm{x} - \bm{x}_j(t))}{|\bm{x} - \bm{x}_i(t)|^3|\bm{x} - \bm{x}_j(t)|^3} \\
	H_{\text{em}\parallel} &= \sum_{i\neq j}\frac{q_i q_j}{8\pi\epsilon_0}\frac{1}{|\bm{x}_i-\bm{x}_j|}
\end{align*}
Which is simply the Coulomb interaction between particles.

So our total Hamiltonian becomes:
\begin{align*}
	H &= \sum_j \frac{1}{2}m_j v_j^2 + H_{\text{em}\parallel} + H_{\text{em}\perp}
\end{align*}
To quantize this we want to find canonically conjugate variables. We know we can write $H$ as:
\begin{align*}
	H &= \sum_j \frac{1}{2m_j} \left[\bm{p}_j - q_j \bm{A}(\bm{x_j},t)\right]^2 + H_{\text{em}\parallel} + H_{\text{em}\perp} + \sum_j q_j \phi(\bm{x_j},t)
\end{align*}
We can write:
\begin{align*}
	H_{\text{em}\perp} &= \sum_{\bm{k},\sigma} \frac{\hbar\omega_k}{2} (Q^2_{\bm{k},\sigma} + P_{\bm{k},\sigma}^2)
\end{align*}
And additionally:
\begin{align*}
	H &= \left[\sum_j \frac{p_j^2}{2m_j} + H_{\text{em}\parallel}\right] + H_{\text{em}\perp} + \sum_j \left[-\frac{q_j}{m_j} \bm{p}_j\cdot\bm{A}(\bm{x},t) + \frac{q^2_j}{2m_j} \bm{A}\cdot\bm{A} + q_j \phi(\bm{x}_j,t)\right] \\
	H_I &= \sum_j \left[-\frac{q_j}{m_j} \bm{p}_j\cdot\bm{A}(\bm{x},t) + \frac{q^2_j}{2m_j} \bm{A}\cdot\bm{A} + q_j \phi(\bm{x}_j,t)\right]
\end{align*}
Where $H_I$ is our interaction term. Here we've split our Hamiltonian into a matter field, a radiation field and an interaction field. We now make the long wavelength (dipole approximation).
This corresponds to saying that $\bm{A}(\bm{x}_j,t) \approx \bm{A}(t)$, for the light matter interaction. This could be consider a ``low energy approximation'' (i.e. we don't properly consider the case of gamma rays here).

We now do a gauge transformation. 
\begin{align*}
	\bm{A}' &= \bm{A} + \del\chi \\
	\phi' &= \phi - \partial_t \chi
\end{align*}
We are free to choose a gauge function:
\begin{align*}
	\chi(\bm{x},t) &= -\bm{x}\cdot\bm{A}(t) \\
	\del\chi &= -\bm{A} \\
	\partial_t \chi &= \bm{x}\cdot\bm{E}_\perp 
	\bm{A}' &= 0 \\
	\phi' &= \phi - \bm{x}\cdot\bm{E}_\perp \\
	\bm{A} &= -\del\chi
\end{align*}
So we can then say:
\begin{align*}
	H_I &= -\sum_j q_j\bm{x}_j\cdot\bm{E}
\end{align*}
Which we often write in terms of our dipole($\bm{d}_j = q_j\bm{x}_k$):
\begin{align*}
	H_I &= -\sum_j \bm{d}_j\cdot\bm{E}
\end{align*}
So then:
\begin{align*}
	H_p &= \sum_j \frac{p_j^2}{2m_j} + H_{\text{em}\parallel} \\
	H_R &= H_{\text{em}\perp} \\
	H &= H_p + H_R + H_I 
\end{align*}
We can think of $H_p$ as the hamiltonian for matter (think hydrogen atom from introductory quantum), $H_R$ is the hamiltonian for the fields like we dealt with last term, and $H_I$ allows us to exchange energy between the atom and fields.
