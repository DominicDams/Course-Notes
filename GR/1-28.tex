We now look to compute the travel time through a wormhole geometry as an example. We have our traveler begin at a coordinate radius $R$ and allow them to fall inward through the Wormhole.
Therefore we say:
\begin{align*}
	u^r &= U
\end{align*}
So we have an initial radial 4-velocity. We now attempt to compute the time the traveler experiences traveling to $r=-R$. In order to have the total length of our 4-velocity equal to $-1$ we say:
\begin{align*}
	\bm{u} &= (\sqrt{1+ U^2},U,0,0)
\end{align*}
We can see from our prior work in this geometry:
\begin{align*}
	\partder{u^r}{\tau} &= 0
\end{align*}
So moving along the radius we see that the radial component is constant and therefore we can say:
\begin{align*}
	r(\tau) &= U\tau \\
	\Delta\tau &= \frac{2R}{U}
\end{align*}
Recall from mechanics, that conservation laws give first integrals of EOM, which generally reduce the number of equations we need to solve.
In GR we know that the four velocity's length is always conserved $g_{\alpha\beta} u^\alpha u^\beta = -1$.
Symmetries in our spacetime will give us other conservation laws. From Newtonian mechanics we know that energy conservation arises from a potential that is invariant under time translation.
Conserved linear momentum arrises from translation invariance. For angular momentum we see a conservation law will arise from rotational invariance (spherical symmetry).

In GR we look at symmetries in spacetime. These occur when the metric is invariant under displacements in a given coordinate.
The four vector $\xi = (0,1,0,0)$ which captures our symmetry is called the Killing vector associated with the symmetry.

For example in flat space:
\begin{align*}
	dS^2 &= dx^2 + dy^2 + dz^2
\end{align*}
We can immediately see that we have killing vectors $(1,0,0)$, $(0,1,0)$, and $(0,0,1)$. We can see if we change coordinate to spherical coordinates we have:
\begin{align*}
	dS^2 &= dr^2 + r^2 d\theta^2 + r^2\sin^2\theta d\phi^2
\end{align*}
So we can see we have the killing vector $(0,0,1)$.

These killing vectors are associated with constants of motion: 
\begin{align*}
	\bm{\xi}\cdot\bm{u} &= \text{const}
\end{align*}
If our metric (and therefore our Lagrangian) are independant of some coordinate $x^1$, then:
\begin{align*}
	\partder{\mathcal{L}}{x^1} &=0 \\
	\frac{d}{d\sigma} \partder{\mathcal{L}}{\frac{dx^1}{d\sigma}} &= 0 \\
	\partder{\mathcal{L}}{\frac{dx^1}{d\sigma}} &= -g_{1\beta} \frac{1}{\mathcal{L}} \frac{dx^\beta}{d\sigma} \\
	\partder{\mathcal{L}}{\frac{dx^1}{d\sigma}} &= -g_{1\beta} \frac{dx^\beta}{d\tau} \\
	\partder{\mathcal{L}}{\frac{dx^1}{d\sigma}} &= -g_{\alpha\beta} \xi^\alpha\frac{dx^\beta}{d\tau} \\
	\partder{\mathcal{L}}{\frac{dx^1}{d\sigma}} &= -\bm{\xi}\cdot\bm{u}
\end{align*}
Looking now at null geodesics (the paths of photons). We can no longer use proper time to parameterize our paths since photons have $\Delta\tau = 0$. We choose our parameter to be $\lambda$, so:
\begin{align*}
	u^\alpha &= \frac{dx^\alpha}{d\lambda} \\
	\bm{u}\cdot\bm{u} &= g_{\alpha\beta} \frac{dx^\alpha}{d\lambda} \frac{dx^\beta}{d\lambda} \\
	\bm{u}\cdot\bm{u} &= 0
\end{align*}
We know in flat spacetime that:
\begin{align*}
	\frac{d^2x^\alpha}{d\lambda^2} &= 0
\end{align*}
We want to generalize this, and we want to be sure to maintain that: 1. in a local inertial frame our generalization reduces to this, and 2. that this takes the same form in all coordinate systems.

We choose as our generalization:
\begin{align*}
	\frac{d^2x^\alpha}{d\lambda^2} &= -\Gamma^\alpha_{\beta\gamma} \frac{dx^\beta}{d\lambda}\frac{dx^\gamma}{d\lambda}
\end{align*}

\subsection{Shwarzchild geometry}
We consider the geometry outside a spherical star:
\begin{align*}
	ds^2 &= -\left(1 -\frac{2GM}{c^2 r}\right)(cdt)^2 + \left(1- \frac{2GM}{c^2 r}\right)^{-1}dr^2 + r^2 (d\theta^2 + \sin^2\theta d\phi^2)
\end{align*}
We refer to these coordinates as Schwarzchild coordinates. We can immediately see this is spherically symmetric, and that as $r\to\infty$ this becomes flat spacetime.
Additionally we see there are coordinate singularities at $r=0$ and $r=\frac{2GM}{c^2}$. The singularity at $r=0$ can be ignored for physical reasons, and the other one is because of the choice of coordinates, and can be removed by changing coordinates.
Also note that $r$ is not radial distance, but instead related to the surface area of a sphere of that radius. Also this is time independant. We can write killing vector:
\begin{align*}
	\xi &= (1,0,0,0)
\end{align*}
And the geometry at constant $r$ and $t$:
\begin{align*}
	d\Sigma^2 &= r^2(d\theta^2 + \sin^2\theta d\phi^2)
\end{align*}
Where we find another killing vector:
\begin{align*}
	\eta &= (0,0,0,1)
\end{align*}
We may assume that $M$ is the mass of the star, but we now want to interogate this by taking the non-relativistic limit:
\begin{align*}
	\frac{GM}{c^2 r} \ll 1 \\
	ds^2 &\approx -\left(1 - \frac{2GM}{c^2r}\right)(cdt)^2 + \left(1+ \frac{2GM}{c^2 r}\right)dr^2 + r^2 (d\theta^2 + \sin^2\theta d\phi^2)
\end{align*}
Which is the static weak field metric for the Newtonian gravitational potential

Looking at the Schwarzchild radius $r = \frac{2GM}{c^2}$.

Now we look at the effect on light of killing vectors moving away from the singularity of a shwarzchild geometry.
If we have an emission event at fixed radius R emitting a light signal with frequency $\omega_*$ and an infinitely distant observer sees it with frequence $\omega_\infty$.
We know the energy of this photon is $E=\hbar\omega$. And we have a symmetry related to this:
\begin{align*}
	\xi &= (1,0,0,0) \\
	\bm{\xi}\cdot\bm{p} &= \text{const}
\end{align*}
We want to move to orthonormal basis for our observers, in these frames we will have each observers velocity vector be $u = (1,0,0,0)$.

Therefore we say an observer moving with velocity $\bm{u}_\text{obs}$ will measure an energy:
\begin{align*}
	E &= -\bm{p}\cdot\bm{u}_\text{obs} \\
	E &= \hbar\omega
\end{align*}
For any observer we will have four velocity:
\begin{align*}
	\bm{u}_\text{obs}\cdot\bm{u}_\text{obs} &= -1
\end{align*}
In the observers frame:
\begin{align*}
	u_\text{obs}^i &= 0 \\
	g_{tt}(r) u^t_\text{obs}\ ^2 &= -1 \\
	u^t_\text{obs} &= \left(1- \frac{2GM}{c^2 r}\right)^{-\frac{1}{2}}
\end{align*}
So for our stationary observer:
\begin{align*}
	\bm{u}_\text{obs} &= \left(1- \frac{2GM}{c^2 r}\right)^{-\frac{1}{2}}\bm{\xi}
\end{align*}
So:
\begin{align*}
	\hbar\omega_* &= \left(1- \frac{2GM}{c^2 r}\right)^{-\frac{1}{2}}(\bm{\xi}\cdot\bm{p})_R \\
	\hbar\omega_\infty &= (\bm{\xi}\cdot\bm{p})_\infty
\end{align*}
But this is a conserved quantity, so:
\begin{align*}
	(\bm{\xi}\cdot\bm{p})_R &= (\bm{\xi}\cdot\bm{p})_\infty \\
	\omega_\infty &= \omega_* \sqrt{1 - \frac{2GM}{c^2 R}}
\end{align*}
